\documentclass{beamer}

\everymath{\displaystyle}

\DeclareMathOperator{\arccot}{arccot}

\newcounter{question}
\newcommand{\correct}{\only<2>{\color{black!30!green}}}
\newcommand{\incorrect}{\only<2>{\color{red}}}
\newcommand{\question}[6]
{\stepcounter{question}
  \begin{frame}
    \frametitle{Question \thequestion}
    #1
    \begin{enumerate}
    \item {\incorrect #2}
    \item {\incorrect #3}
    \item {\incorrect #4}
    \item {\incorrect #5}
    \item {\incorrect #6}
    \end{enumerate}
  \end{frame}}

\begin{document}

\renewcommand{\theenumi}{\Alph{enumi}}

\begin{frame}
  \title{Calculus Bowl: Expert Edition}
  \author{
    Radon Rosborough \\
    \hfill \\
    \href
    {https://intuitiveexplanations.com/math/calculus-bowl-expert-edition/}
    {https://intuitiveexplanations.com/math/calculus-bowl-expert-edition/}}
  \date{April 2016}
  \maketitle
\end{frame}

% CB:BEGIN

\question
{For what strictly positive number $x$ is $x^x$ the smallest?}
% CB:IGNORE
{\correct $1/e$}
{$\ln 2$}
{$1$}
{$e$}
{There is no such $x$}

\question
{Find the value of $\sum_{k=0}^\infty e^{-\pi k} \cos(-\pi k)$.}
{$\frac{1}{1 + e^\pi}$}
{$\frac{1}{1 - e^\pi}$}
{\correct $\frac{e^\pi}{1 + e^\pi}$}
{$\frac{e^{(\pi^2)}}{1 - e^{(\pi^2)}}$}
{$\frac{e^{(\pi^2)}}{e^{(\pi^2)} - 1}$}

\question
{What is the area in Quadrant IV above the curve $y = \ln x$?}
% CB:IGNORE
{\correct $1$}
{$\ln 2$}
{$e - 1$}
{$e$}
{$\infty$}

\question
{Find the slope of the tangent line to $y = \frac{9x^3 - 27x^2 + 5x -
    6}{55x^4 + 11x^3 - 7x^2 + 6x + 3}$ at $x = 0$.}
% CB:IGNORE
{$1$}
{\correct $2$}
{$3$}
{$4$}
{$6$}

\question
{Find $\lim_{x \to \infty} \sin(\arctan x)$.}
% CB:IGNORE
{$-1$}
{$0$}
{$1/2$}
{\correct $1$}
{The limit does not exist}

\question
{Let $f(x) = \begin{cases}
    mx + b & x \leq 1 \\
    x^2 & x > 1
  \end{cases}$ . If $f$ is everywhere differentiable, what are $m$ and $b$?}
{$m = -2$, $b = -1$}
{$m = 3$, $b = -2$}
{\correct $m = 2$, $b = -1$}
{$m = 1$, $b = 0$}
{$m = 2$, $b = -2$}

\question
{If $x = y^2$, $y = z^3$, and $z = w^4$, then what is $\frac{dx}{dy} +
  \frac{dx}{dz} + \frac{dx}{dw}$?}
{\correct $2w^{12} + 6w^{20} + 24w^{23}$}
{$2w^8 + 3w^6 + 4w^3$}
{$2w^{12} + 3w^8 + 4w^3$}
{$2w + 3w^2 + 4w^3$}
{$2w + 6w^5 + 24w^{23}$}

\question
{Suppose that $t$ is measured in meters and $v(t)$ is measured in
  seconds. What are the units of $\int v''(t) \,dt$?}
{s}
{m}
{m/s$^2$}
{m/s}
{\correct s/m}

\question
{Let $f$ be a continuous function such that $f(x) + f(1 - x) \neq 0$
  for all $x$. Evaluate $\int_0^1 \frac{f(x)}{f(x) + f(1 - x)} \,dx$.}
{$0$}
{Cannot be determined without more information}
{\correct $1/2$}
{$1$}
{$2$}

\question
{Suppose that $\int_{-3}^4 f(x) \,dx = 3$, $\int_{-1}^7 f(x) \,dx =
  7$, and $\int_{-3}^7 f(x) \,dx = 5$. What is $\int_{-1}^4 f(x)
  \,dx$?}
{\correct $5$}
{$8$}
{$2$}
{$10$}
{$-5$}

\question
{Find the value of $\sum_{n=0}^\infty
  \frac{\sin(\frac{n\pi}{2})}{n!}$.}
% CB:IGNORE
{$\cos 1$}
{$\ln 2$}
{\correct $\sin 1$}
{$1$}
{$e$}

\question
{If $x^2 + y^2 = 1$, then what is $\frac{d^2y}{dx^2}$?}
{$\frac{y^3}{x^2 - y^2}$}
{\correct $-\frac{1}{y^3}$}
{$-\frac{x}{y}$}
{$-\frac{3x}{y^5}$}
{$\frac{y^2 - x^2}{y^3}$}

\question
{Which of the following is \alert{not} a correct solution to the
  problem $\int 2 \tan x \sec^2 x \,dx$?}
{$\frac{2}{1 + \cos 2x}$}
{$\sec^2 x$}
{\correct $\sqrt{\frac{1 - \cos x}{1 + \cos x}}$}
{$\frac{4}{(e^{ix} + e^{-ix})^2}$}
{$\frac{1 - \cos 2x}{1 + \cos 2x}$}

\question
{Rolle's theorem is a special case of which of the following theorems?}
{Extreme value theorem}
{\correct Mean value theorem}
{Fundamental theorem of calculus}
{Intermediate value theorem}
{Mean value theorem for integrals}

\question
{Suppose that the position of a particle is given by the equation $x =
  \sin(\pi t)$. Find its velocity when $t = 1/6$, given that $x$ is
  time and $t$ is position.}
{$\frac{\sqrt{3} \pi}{2}$}
{\correct $\frac{2}{\sqrt{3} \pi}$}
{$\frac{4}{3 \sqrt{3} \pi}$}
{$\frac{6}{\sqrt{35} \pi}$}
{$0$}

\question
{What is the behavior of the expression $\lim_{m \to \infty} \lim_{n
    \to \infty} \cos^{2n}(m! \pi x)$? Assume $m$ and $n$ are
  integers.}
% CB:IGNORE
{It is always zero}
{It is always one}
{It does not exist}
{\correct It is zero if $x$ is irrational and one if $x$ is rational}
{It is zero if $x$ is not an integer and one if $x$ is an integer}

\question
{Identify the differential equations whose solutions would be
  underestimated by Euler's method.
  {\renewcommand{\theenumi}{\Roman{enumi}}
    \begin{enumerate}
    \item $\frac{dx}{dt} = 1$
    \item $\frac{dx}{dt} = t$
    \item $\frac{dx}{dt} = x$
    \end{enumerate}}}
% CB:IGNORE
{II only}
{III only}
{I and II only}
{\correct II and III only}
{I, II, and III}

\question
{Find $\int_\pi^{\pi^2} \frac{\pi^2 \ln(\pi^\pi)}{\theta(\pi + \pi^2)}
  \,d\theta$.}
{$\frac{2\pi(\ln \pi)^2}{1 + \pi}$}
{$\frac{\pi^2 \ln \pi}{1 + \pi}$}
{$\frac{2\pi^2(\ln \pi)^2}{1 + \pi}$}
{\correct $\frac{\pi^2(\ln \pi)^2}{1 + \pi}$}
{$\frac{\pi^3(\pi - 1)\ln \pi}{\pi + 1}$}

\question
{Determine the surface area of the solid generated by revolving the
  curve $y = \sqrt{4 - x^2}$ about the $x$-axis.}
% CB:IGNORE
{$2 \pi$}
{$4 \pi$}
{$\frac{16}{3} \pi$}
{$8 \pi$}
{\correct $16 \pi$}

\question
{Which of the following integrals, when evaluated, gives the area of
  one of the regions bounded by the curves $y = \cos x$ and $y = \sin
  x$?}
{$\int_{\pi/4}^{5\pi/4} \cos x - \sin x \,dx$}
{$\int_0^{\pi/4} \cos x - \sin x \,dx$}
{\correct $\int_{\pi/4}^{-3\pi/4} \sin x - \cos x \,dx$}
{$\int_0^\pi |\cos x| - |\sin x| \,dx$}
{$\int_0^\pi \cos x - \sin x \,dx$}

\question
{What is the $k$th derivative of $x^n$?}
{$k! x^{n-k}$}
{$\frac{n!}{(n - k - 1)!} x^{n-k}$}
{\correct $\frac{n!}{(n - k)!} x^{n-k}$}
{$n! x^{n-k}$}
{$(n - k)! x^{n-k}$}

\question
{What is the average value of the function $f(x) = \sqrt{1 - x^2}$
  over its domain?}
{\correct $\pi/4$}
{$\pi/6$}
{$3/4$}
{$\sqrt{3}/2$}
{$5/8$}

\question
{Evaluate $\lim_{x \to \infty} \frac{\sum_{n=3}^{9}
    nx^n}{\sqrt[3]{\sum_{n=9}^{27} \frac{x^n}{n}}}$.}
% CB:IGNORE
{$0$}
{$3$}
{$9$}
{\correct $27$}
{$\infty$}

\question
{Evaluate $\frac{d}{dx} \big[\ln \ln \ln \ln x\big]$ at $x =
  e^{(e^e)}$.}
{$e^{-e - 1}$}
{$e^{e^e + e + 1}$}
{\correct $e^{-e^e - e - 1}$}
{$e^{-e^{-e - 1} - e}$}
{$e^{-e^2 - e - 1}$}

\question
{Find the value of $\lim_{x \to 0} \frac{\ln(1 - x) - \sin x}{1 - \cos^2 x}$.}
{\correct The limit does not exist!}
{$\infty$}
{$1/2$}
{$-\infty$}
{$0$}

\question
{Taken collectively, how many different intervals of convergence do
  the following series have?
  \begin{align*}
    &\sum_{n=1}^\infty r^n &&\sum_{n=1}^\infty n^{-r} &&\sum_{n=1}^\infty r^{-n} \\
    &\sum_{n=1}^\infty n^r &&\sum_{n=1}^\infty (-r)^n &&\sum_{n=1}^\infty (-r)^{-n}
  \end{align*}}
% CB:IGNORE
{$1$}
{$2$}
{$3$}
{\correct $4$}
{$6$}

\question
{How many points of inflection can a polynomial of degree $7$ have, at
  most?}
% CB:IGNORE
{$3$}
{\correct $5$}
{$6$}
{$7$}
{$8$}

\question
{Evaluate $\frac{d}{dx} \left[(\ln x)^{\ln x}\right]$.}
{$(\ln x)^{\ln x} \left(\frac{\ln x + \ln \ln x}{x}\right)$}
{\correct $(\ln x)^{\ln x} \left(\frac{1 + \ln \ln x}{x}\right)$}
{$(\ln x)^{\ln x} \left(\frac{1 + \ln x}{\ln \ln x}\right)$}
{$(\ln x)^{\ln x} \left(1 + \frac{\ln \ln x}{x}\right)$}
{$(\ln x)^{\ln x} \left(\frac{1 + \ln \ln x}{\ln x}\right)$}

\question
{Evaluate $\int_{-\infty}^0 x^5 e^x \,dx$.}
% CB:IGNORE
{\correct $-120$}
{$-24$}
{$-5$}
{$-1$}
{The integral cannot be expressed in closed form}

\question
{The power rule states that $\frac{d}{dx} \big[x^n\big] = nx^{n-1}$.
  In which of the following situations does the power rule \alert{not}
  hold?
  {\renewcommand{\theenumi}{\Roman{enumi}}
    \begin{enumerate}
    \item $x = 0$ and $n < 1$
    \item $x = 0$ and $n = 1$
    \item $x < 0$ and $n$ is irrational
    \end{enumerate}}}
% CB:IGNORE
{None}
{I only}
{II only}
{I and II only}
{\correct I, II, and III}

\question
{Compute $\int_0^\infty (e^{-x})^2 \,dx$.}
% CB:IGNORE
{\correct $1/2$}
{$\sqrt{\pi}/2$}
{$1$}
{$\sqrt{\pi}$}
{The integral does not converge}

\question
{If the arc length of the curve given by $x(t) = \cos \sqrt{t}$ and
  $y(t) = \sin \sqrt{t}$ from $t = 0$ to $t = T$ is $6\pi$, then what
  is $T$?}
% CB:IGNORE
{$\sqrt{3}$}
{$\sqrt{6}$}
{$3$}
{$6$}
{\correct $9$}

\question
{Evaluate $\frac{d}{dx} \Big[x + x^2 + x^3 + \cdots + x^{100}\Big]$ at
  $x = 1$.}
% CB:IGNORE
{$99$}
{$100$}
{\correct $5050$}
{$10100$}
{$100!$}

\question
{What is the rate of change of the length of the hypotenuse of an
  isosceles right triangle with respect to the length of one of its
  legs?}
% CB:IGNORE
{$0$}
{$1$}
{\correct $\sqrt{2}$}
{$2$}
{$2 \sqrt{2}$}

\question
{What is $\frac{d}{dA} \big[ABRACADABRA\big]$ when $A = 1$, $B = 2$,
  $C = 3$, and $D = 4$?}
% CB:IGNORE
{$5R^2$}
{$48R^2$}
{$120R^2$}
{$192R^2$}
{\correct $240R^2$}

\question
{Determine the value of $\frac{d}{dx} \Big[|\sin x| + |\cos x| + |\tan
  x|\Big]$ at $x = \frac{\pi}{4}$.}
% CB:IGNORE
{$1$}
{$\sqrt{2}$}
{$1 + \frac{\sqrt{2}}{2}$}
{\correct $2$}
{$1 + \sqrt{2}$}

\question
{Compute $\frac{d}{dx} \left[\frac{x}{\ln x}\right]$.}
{$\frac{1}{\ln x} + \frac{1}{(\ln x)^2}$}
{\correct $\frac{1}{\ln x} - \frac{1}{(\ln x)^2}$}
{$\frac{1}{x^2} - \frac{\ln x}{x^2}$}
{$\frac{1}{(\ln x)^2} - \frac{1}{\ln x}$}
{$\frac{\ln x - 1}{\ln(x^2)}$}

\question
{Suppose that the radius of a sphere is $1$ inch, and its volume is
  increasing at a rate of $1$ cubic inch per minute. How fast is its
  surface area increasing?}
% CB:IGNORE
{$1$ in$^2$/min}
{\correct $2$ in$^2$/min}
{$2\pi$ in$^2$/min}
{$4\pi$ in$^2$/min}
{$8\pi$ in$^2$/min}

\question
{What is the average value of $\arctan x$ from $x = 0$ to $x = 1$?}
{$\frac{\pi}{4} - \frac{1}{2}$}
{\correct $\frac{\pi}{4} - \ln \sqrt{2}$}
{$\frac{1}{4} + \ln 2$}
{$\frac{\pi}{2} - \frac{3}{4}$}
{$\ln\left(\frac{\pi}{2}\right)$}

\question
{Find the average value of $\tan x$ for $x$ in the interval $[0, \pi/4]$.}
{$\frac{1}{2}$}
{$\frac{2}{\pi} + \frac{1}{4}$}
{$1 - \frac{2}{\pi}$}
{$\frac{\pi}{4}$}
{\correct $\frac{4}{\pi} - 1$}

\question
{Find $\lim_{x \to 0} |x|^x$.}
% CB:IGNORE
{$0$}
{$1/2$}
{\correct $1$}
{$\infty$}
{The limit does not exist}

\question
{Evaluate $\frac{d}{dx} \int_x^{2x} \frac{x}{t} \,dt$.}
{$2x$}
{$\ln x$}
{\correct $\ln 2$}
{$1$}
{$x$}

\question
{Find $\int_{-1}^2 \frac{dx}{x}$.}
% CB:IGNORE
{$0$}
{$\ln 2$}
{$1$}
{$e$}
{\correct The integral does not converge}

\question
{For what value of $p$ is the expression $\lim_{n \to \infty}
  \frac{\sqrt{1} + \sqrt{2} + \sqrt{3} + \cdots + \sqrt{n}}{n^p}$
  finite and nonzero?}
% CB:IGNORE
{$1/2$}
{$1$}
{\correct $3/2$}
{$2$}
{$5/2$}

\question
{Suppose that $x^y = y^x$. Find $\frac{dy}{dx}$.}
{$\frac{x^2 \ln y - xy}{y^2 \ln x - xy}$}
{\correct $\frac{xy \ln y - y^2}{xy \ln x - x^2}$}
{$\frac{xy \ln y - x^2}{xy \ln x - y^2}$}
{$\frac{x^2 \ln x - y^2}{y^2 \ln y - x^2}$}
{$\frac{xy \ln x - y^2}{xy \ln y - x^2}$}

\question
{Consider a collection of spheres with radii given by $\left\{1,
    \frac{1}{2^p}, \frac{1}{3^p}, \ldots\right\}$. If the total volume
  is finite but the total surface area is infinite, then what values
  of $p$ are possible?}
{$2 \leq p < 3$}
{\correct $\frac{1}{3} < p \leq \frac{1}{2}$}
{$2 < p \leq 3$}
{$\frac{1}{3} < p \leq 1$}
{$1 < p \leq 3$}

\question
{Determine the value of $\sqrt{1 + \sqrt{1 + \sqrt{1 + \cdots}}}$ .}
{$\frac{\sqrt{2}}{5}$}
{$\frac{1 - \sqrt{5}}{2}$}
{$\frac{1 - \sqrt{3}}{2}$}
{\correct $\frac{1 + \sqrt{5}}{2}$}
{$\frac{1 + \sqrt{3}}{2}$}

\question
{Determine $\frac{d}{dx} \Big[(\ln x)^e\Big]$.}
{\correct $\frac{e}{x} (\ln x)^{e-1}$}
{$(\ln \ln x) (\ln x)^e$}
{$\frac{e}{x}$}
{$(\ln x)^e \left(1 + \frac{e}{x}\right)$}
{$1$}

\question
{Suppose that $f$ is a function defined on an interval $I$. Which of
  the following statements \alert{must} be true?
  {\renewcommand{\theenumi}{\Roman{enumi}}
    \begin{enumerate}
    \item If $f$ is continuous on $I$, then $f$ is differentiable on $I$.
    \item If $f$ is differentiable on $I$, then $f$ is continuous on $I$.
    \item If $f$ is integrable on $I$, then $f$ is continuous on $I$.
    \item If $f$ is continuous on $I$, then $f$ is integrable on $I$.
    \end{enumerate}}}
% CB:IGNORE
{II only}
{I and III only}
{\correct II and IV only}
{II, III, and IV only}
{I, II, III, and IV}

% CB:END

{}{}{}{}{}

\end{document}
