\documentclass{myart}
\usepackage{etoolbox}

\allowdisplaybreaks

\newcommand{\term}[1]{\textbf{#1}}
\newcommand{\eq}[1]{(\ref{eq:#1})}

\newcommand{\deriv}[3][]{\frac{d^{#1}#2}{d#3^{#1}}}
\newcommand{\pderiv}[3][]{\frac{\partial^{#1}#2}{\partial#3^{#1}}}
\newcommand{\ppderiv}[5]{
  \frac{\partial^{
        \the\numexpr \ifstrempty{#3}{1}{#3}
      + \ifstrempty{#5}{1}{#5}
        \relax}
      #1}
    {\partial#4^{#5} \partial#2^{#3}}}
\newcommand{\fderiv}[3][]{d^{#1}#2/d#3^{#1}}
\newcommand{\fpderiv}[3][]{\partial^{#1}#2/\partial#3^{#1}}
\DeclareMathOperator{\OpRe}{Re}
\DeclareMathOperator{\OpIm}{Im}
\renewcommand{\Re}[1]{\OpRe\left\{#1\right\}}
\renewcommand{\Im}[1]{\OpIm\left\{#1\right\}}
\renewcommand{\L}[1]{\mathcal{L}\left\{#1\right\}}
\newcommand{\Li}[1]{\mathcal{L}^{-1}\left\{#1\right\}}
\renewcommand{\vec}[1]{\mathbf{#1}}
\newcommand{\vect}[1]{\left\langle#1\right\rangle}
\newcommand{\Vect}[2]{\begin{bmatrix*} #1 \\ #2 \end{bmatrix*}}
\newcommand{\Vector}[3]{\begin{bmatrix*} #1 \\ #2 \\ #3 \end{bmatrix*}}
\newcommand{\mat}[2][r]{\begin{bmatrix*}[#1] #2 \end{bmatrix*}}
\newcommand{\cvdots}[1][=]{\mathrel{\makebox[\widthof{#1}]{\vdots}}}
\newcommand{\dmat}[2][r]{\begin{vmatrix*}[#1] #2 \end{vmatrix*}}
\newcommand{\W}[2][]{W_{#1}\left[#2\right]}

\begin{document}

\titlepage
{Differential Equation Solution Strategies}
{differential-equation-solution-strategies}
{This document is a compilation of various strategies for solving
  differential equations. Proofs or derivations are presented for most
  of the methods discussed. The relevant definitions and properties
  concerning the various types of differential equations are also
  presented. At the end of each section, there is a subsection that
  explains when the different methods are usually applicable. Many of
  the methods presented in this document are based on the textbook I
  used in my Differential Equations class (\textit{Differential
    Equations and Their Applications} by Martin Braun, fourth
  edition).}

\tableofcontents

\section{First-Order Differential Equations}
\label{sec:first order}

\subsection{Characterization of First-Order Equations}
\label{subsec:first order characterization}

A \term{first-order differential equation} is an equation of the form
\begin{equation*}
  \deriv{y}{t} = f(t, y).
\end{equation*}
If an additional equation of the form
\begin{equation*}
  y(t_0) = y_0
\end{equation*}
is also given, then the two equations together are called an
\term{initial-value problem}. Every initial-value problem has a unique
solution; if an initial condition is not provided, the general
solution will have one arbitrary constant.\footnote{This requires that
  $f$ be sufficiently well-behaved.} A first-order differential
equation of the form
\begin{equation} \label{eq:first order nonhomogeneous}
  \deriv{y}{t} + p(t)y = q(t)
\end{equation}
is called \term{linear}. If $q(t) = 0$ then the equation is of the
form
\begin{equation} \label{eq:first order homogeneous}
  \deriv{y}{t} + p(t)y = 0
\end{equation}
and it is called \term{homogeneous}; otherwise, it is called
\term{nonhomogeneous}.

If $y_1(t)$ is a solution of \eq{first order homogeneous} that is not
always equal to zero, then the general solution to \eq{first order
  homogeneous} is
\begin{equation*}
  y(t) = Cy_1(t).
  \footnote{This requires that $p$ be sufficiently well-behaved.}
\end{equation*}
If $y_1(t)$ is a solution of \eq{first order homogeneous} that is not
always equal to zero and $\psi(t)$ is a solution of \eq{first order
  nonhomogeneous}, then the general solution to \eq{first order
  nonhomogeneous} is
\begin{equation*}
  y(t) = Cy_1(t) + \psi(t).
  \footnote{This requires that $p$ and $q$ be sufficiently
    well-behaved.}
\end{equation*}

\subsection{Direct Integration}
\label{subsec:direct integration}

The simplest possible differential equation is of the form
\begin{equation*}
  \deriv{y}{t} = f(t).
\end{equation*}
Integrating both sides with respect to $t$ gives the general solution
\begin{equation*}
  y(t) = \int f(t) \,dt + C.
\end{equation*}
If we are given the initial condition $y(t_0) = y_0$, then we may find
a particular solution immediately, without first finding the general
solution, in the following manner:
\begin{align*}
  \int_{t_0}^t y(t) \,dt &= \int_{t_0}^t f(t) \,dt \\
  y(t) - y_0 &= \int_{t_0}^t f(t) \,dt \\
  y(t) &= y_0 + \int_{t_0}^t f(t) \,dt.
\end{align*}

\subsection{Homogeneous Linear Equations}
\label{subsec:first order homogeneous}

To solve the general homogeneous first-order linear differential
equation, we may proceed in the following manner:
\begin{align*}
  \deriv{y}{t} + p(t)y &= 0 \\
  \deriv{y}{t} &= -p(t)y \\
  \frac{\fderiv{y}{t}}{y} &= -p(t) \\
  \int \frac{\fderiv{y}{t}}{y} \,dt &= -\int p(t) \,dt + C \\
  \int \frac{dy}{y} &= -\int p(t) \,dt + C \\
  \ln |y| &= -\int p(t) \,dt + C \\
  |y| &= e^{-\int p(t) \,dt + C} \\
  |y| &= Ce^{-\int p(t) \,dt}. \\
\end{align*}
Now, $y$ must have the same sign for all $t$, because otherwise the
integral on the left-hand side would cross the singularity at $y = 0$.
Consequently, by adjusting the value of $C$ appropriately, we may
still eliminate the absolute value bars from $y$. The general solution
is then
\begin{equation*}
  y = Ce^{-\int p(t) \,dt}.
\end{equation*}
If we are given the initial condition $y(t_0) = y_0$, then we may find
a particular solution immediately, without first finding the general
solution, in the following manner:
\begin{align*}
  \deriv{y}{t} + p(t)y &= 0 \\
  \frac{\fderiv{y}{t}}{y} &= -p(t) \\
  \int_{t_0}^t \frac{\fderiv{y}{t}}{y} \,dt &= -\int_{t_0}^t p(t) \,dt \\
  \int_{y_0}^y \frac{dy}{y} &= -\int_{t_0}^t p(t) \,dt \\
  \ln \left|\frac{y}{y_0}\right| &= -\int_{t_0}^t p(t) \,dt \\
  \left|\frac{y}{y_0}\right| &= e^{-\int_{t_0}^t p(t) \,dt}.
\end{align*}
Now, $y_0$ and $y$ must have the same sign, because otherwise the
integral on the left-hand side would cross the singularity at $y = 0$.
Thus the quantity $y/y_0$ is positive, and since the right-hand side
is also positive it follows that we may eliminate the absolute value
bars. Multiplying by $y_0$ then gives the general solution:
\begin{equation*}
  y = y_0e^{-\int_{t_0}^t p(t) \,dt}.
\end{equation*}

\subsection{Nonhomogeneous Linear Equations}
\label{subsec:first order nonhomogeneous}

We will now solve the general nonhomogeneous first-order linear
differential equation
\begin{equation} \label{eq:first order nonhomogeneous solution}
  \deriv{y}{t} + p(t)y = q(t).
\end{equation}
First, note that
\begin{equation*}
  \deriv{}{t}\Big[ye^{f(t)}\Big] = e^{f(t)}\Big[\deriv{y}{t} + f'(t)y\Big].
\end{equation*}
If we let $f(t) = \int p(t) \,dt$ then we obtain
\begin{equation*}
  \deriv{}{t}\Big[ye^{\int p(t) \,dt}\Big]
  = e^{\int p(t) \,dt}\Big[\deriv{y}{t} + p(t)y\Big].
\end{equation*}
This suggests multiplying both sides of \eq{first order nonhomogeneous
  solution} by the quantity $e^{\int p(t) \,dt}$. Doing so allows us
to proceed in the following manner:
\begin{align*}
     e^{\int p(t) \,dt}\Big[\deriv{y}{t} + p(t)y\Big]
  &= e^{\int p(t) \,dt} q(t) \\
     \deriv{}{t} \Big[ye^{\int p(t) \,dt}\Big]
  &= e^{\int p(t) \,dt} q(t) \\
     ye^{\int p(t) \,dt}
  &= \int e^{\int p(t) \,dt} q(t) \,dt + C \\
     y
  &= e^{-\int p(t) \,dt}
     \left[\int e^{\int p(t) \,dt} q(t) \,dt + C\right].
\end{align*}
If we are given the initial condition $y(t_0) = y_0$, then we may find
a particular solution immediately, without first finding the general
solution, in the following manner:
\begin{align*}
  e^{\int p(t) \,dt}\Big[\deriv{y}{t} + p(t)y\Big]
  &= e^{\int p(t) \,dt} q(t) \\
  \deriv{}{t} \Big[ye^{\int p(t) \,dt}\Big]
  &= e^{\int p(t) \,dt} q(t) \\
  ye^{\int p(t) \,dt} - y_0\left[e^{\int p(t) \,dt}\right]_{t=t_0}
  &= \int_{t_0}^t e^{\int p(t) \,dt} q(t) \,dt \\
  ye^{\left[\int p(t) \,dt\right]_{t=t}}
  &= y_0e^{\left[\int p(t) \,dt\right]_{t=t_0}}
    + \int_{t_0}^t e^{\int p(t) \,dt} q(t) \,dt \\
  y &= y_0e^{\left[\int p(t) \,dt\right]_{t=t_0}
      - \left[\int p(t) \,dt\right]_{t=t}}
      + e^{\int p(t) \,dt} \int_{t_0}^t e^{\int p(t) \,dt} q(t) \,dt \\
  y &= y_0e^{-\int_{t_0}^t p(t) \,dt}
      + e^{-\int p(t) \,dt} \int_{t_0}^t e^{\int p(t) \,dt} q(t) \,dt.
\end{align*}

\subsection{Separable Equations}
\label{subsec:separable}

A first-order differential equation of the form
\begin{equation*}
  \deriv{y}{t} = \frac{g(t)}{f(y)}
\end{equation*}
is called \term{separable}. This equation may be solved easily by
moving the $f(y)$ term to the left-hand side and integrating:
\begin{align*}
  f(y) \deriv{y}{t} &= g(t) \\
  \int f(y) \deriv{y}{t} \,dt &= \int g(t) \,dt \\
  \int f(y) \,dy &= \int g(t) \,dt.
\end{align*}
In general, it is not possible to find an explicit formula for $y$ in
terms of $t$.

If we are given the initial condition $y(t_0) = y_0$, then we may find
a particular solution immediately, without first finding the general
solution, in the following manner:
\begin{align*}
  f(y) \deriv{y}{t} &= g(t) \\
  \int_{t_0}^t f(y) \deriv{y}{t} \,dt &= \int_{t_0}^t g(t) \,dt \\
  \int_{y_0}^y f(y) \,dy &= \int_{t_0}^t g(t) \,dt.
\end{align*}
Again, it is not possible in general to find an explicit formula for
$y$ in terms of $t$.

\subsection{Exact Equations}
\label{subsec:exact}

Consider the first-order differential equation
\begin{equation*}
  f_t(t, y) + f_y(t, y) \deriv{y}{t} = 0,
\end{equation*}
where $f_t$ and $f_y$ are functions of two variables. To solve this
differential equation, observe that
\begin{equation*}
  \deriv{}{t} f(t, y) = \pderiv{f}{t} + \pderiv{f}{y} \deriv{y}{t}
\end{equation*}
according to the multivariable chain rule. Hence, if there exists a
function $f(t, y)$ such that $\fpderiv{f}{t} = f_t$ and
$\fpderiv{f}{y} = f_y$, then we may rewrite the differential equation
as
\begin{equation*}
  \deriv{}{t} f(t, y) = 0.
\end{equation*}
Integrating both sides then easily gives the general solution as
\begin{equation*}
  f(t, y) = C.
\end{equation*}
In general, it is not possible to find an explicit formula for $y$ in
terms of $t$.

Differential equations for which this strategy works---that is, for
which such an $f$ exists---are called \term{exact}.

\subsubsection{Finding a function $f$}
\label{subsubsec:finding f}

From vector analysis, we know that such a function $f$ exists if and
only if
\begin{equation*}
  \pderiv{f_t}{y} = \pderiv{f_y}{t}.
  \footnote{This requires that $f_t$ and $f_y$ be sufficiently
    well-behaved.}
\end{equation*}
If this equation is satisfied, then we may find $f$ in three different
ways. The first way, which is typically the easiest, is to note that
\begin{align*}
  f(t, y) &= \int f_t(t, y) \,dt + C(y) \\
          &= \int f_y(t, y) \,dy + C(t)
\end{align*}
and then determine the functions $C(y)$ and $C(t)$, up to a constant,
by pattern recognition.

The second way, which is useful if the integral $\int f_y(t, y) \,dy$
is difficult to compute, is to start with the first relation from the
first method,
\begin{equation*}
  f(t, y) = \int f_t(t, y) \,dt + C(y),
\end{equation*}
and differentiate both sides with respect to $y$:
\begin{equation*}
  f_y(t, y) = \pderiv{}{y} \int f_t(t, y) \,dt + C'(y).
\end{equation*}
Subtracting the quantity $\pderiv{}{y} \int f_t(t, y) \,dt$ from both
sides gives the equation
\begin{equation*}
  C'(y) = f_y(t, y) - \pderiv{}{y} \int f_t(t, y) \,dt,
\end{equation*}
which may be integrated to find
\begin{equation*}
  C(y) = \int f_y(t, y) - \pderiv{}{y} \int f_t(t, y) \,dt \,dy + C.
\end{equation*}
Thus, we find that
\begin{equation*}
    f(t, y)
  = \int f_t(t, y) \,dt
  + \int f_y(t, y)
  - \pderiv{}{y} \int f_t(t, y) \,dt \,dy
  + C.
\end{equation*}
Any terms of $f_y(t, y)$ containing both $t$ and $y$ will be canceled
by the term $\pderiv{}{y} \int f_t(t, y) \,dt$, which may simplify the
integration. Note also that this formula will give an incorrect answer
if $\fpderiv{f_t}{y} \neq \fpderiv{f_y}{t}$, so check this first.

On the other hand, if the integral $\int f_t(t, y) \,dt$ is difficult
to compute, we may use the third method, which is analogous to the
second. We start with the second relation from the first method,
\begin{equation*}
  f(t, y) = \int f_y(t, y) \,dy + C(t),
\end{equation*}
and differentiate both sides with respect to $t$:
\begin{equation*}
  f_t(t, y) = \pderiv{}{t} \int f_y(t, y) \,dy + C'(t).
\end{equation*}
Subtracting the quantity $\pderiv{}{t} \int f_y(t, y) \,dy$ from both
sides gives the equation
\begin{equation*}
  C'(t) = f_t(t, y) - \pderiv{}{t} \int f_y(t, y) \,dy,
\end{equation*}
which may be integrated to find
\begin{equation*}
  C(t) = \int f_t(t, y) - \pderiv{}{t} \int f_y(t, y) \,dy \,dt.
\end{equation*}
Thus, we find that
\begin{equation*}
  f(t, y) = \int f_y(t, y) \,dy
          + \int f_t(t, y) - \pderiv{}{t} \int f_y(t, y) \,dy \,dt.
\end{equation*}
Any terms of $f_t(t, y)$ containing both $t$ and $y$ will be canceled
by the term $\pderiv{}{t} \int f_y(t, y) \,dt$, which may simplify the
integration. Note also that this formula will give an incorrect answer
if $\fpderiv{f_t}{y} \neq \fpderiv{f_y}{t}$, so check this first.

\subsubsection{Making an equation exact}
\label{subsubsec:making an equation exact}

The obvious shortcoming of our work in the previous section is that if
there is no function $f(t, y)$ such that $\fpderiv{f}{t} = f_t$ and
$\fpderiv{f}{y} = f_y$, then we are stuck. In particular, as we stated
earlier, if
\begin{equation*}
  \pderiv{f_t}{y} \neq \pderiv{f_y}{t},
\end{equation*}
then the differential equation is not exact. However, consider
multiplying both sides of the equation by an arbitrary function
$\mu(t, y)$, to obtain
\begin{equation*}
  f_t(t, y) \mu(t, y) + f_y(t, y) \mu(t, y) \deriv{y}{t} = 0.
\end{equation*}
This differential equation is exact if
\begin{align*}
     \pderiv{}{y} \Big[f_t(t, y) \mu(t, y)\Big]
  &= \pderiv{}{t} \Big[f_y(t, y) \mu(t, y)\Big] \\
     \pderiv{f_t}{y} \mu + f_t \pderiv{\mu}{y}
  &= \pderiv{f_y}{t} \mu + f_y \pderiv{\mu}{t}.
\end{align*}
Unfortunately, this is a partial differential equation that we cannot
solve for $\mu(t, y)$ in general. However, there are two special
cases: if $\mu$ is a function of $t$ alone or a function of $y$ alone.
If we assume that $\mu$ is a function of $t$ alone, then
$\fpderiv{\mu}{y} = 0$ and we obtain
\begin{equation*}
  \pderiv{f_t}{y} \mu = \pderiv{f_y}{t} \mu + f_y \pderiv{\mu}{t},
\end{equation*}
which can be solved for $\fpderiv{\mu}{t}$ to find
\begin{equation*}
  \pderiv{\mu}{t} = \left(\frac{\fpderiv{f_t}{y}
                  - \fpderiv{f_y}{t}}{f_y}\right) \mu.
\end{equation*}
If the right-hand side is not a function of $t$ alone, then this
equation has no solution. If it is, however, then this is a
homogeneous first-order linear differential equation that can be
solved to obtain
\begin{equation*}
  \mu(t) = \exp\left(\int \frac{\fpderiv{f_t}{y}
         - \fpderiv{f_y}{t}}{f_y} \,dt\right).
\end{equation*}
On the other hand, if we assume that $\mu$ is a function of $y$ alone,
then $\fpderiv{\mu}{t} = 0$ and we obtain
\begin{equation*}
  \pderiv{f_t}{y} \mu + f_t \pderiv{\mu}{y} = \pderiv{f_y}{t} \mu,
\end{equation*}
which can be solved for $\fpderiv{\mu}{y}$ to find
\begin{equation*}
  \pderiv{\mu}{y} = \left(\frac{\fpderiv{f_y}{t}
                  - \fpderiv{f_t}{y}}{f_t}\right) \mu.
\end{equation*}
If the right-hand side is not a function of $y$ alone, then this
equation has no solution. If it is, however, then this is a
homogeneous first-order linear differential equation that can be
solved to obtain
\begin{equation*}
  \mu(y) = \exp\left(\frac{\fpderiv{f_y}{t}
         - \fpderiv{f_t}{y}}{f_t}\right).
\end{equation*}
Having solved for $\mu$, we can multiply both sides of the original
differential equation by it to make the equation exact. Then we may
find a function $f$ as before.

\subsection{What to Do in General}
\label{subsec:first order strategy}

The strategies discussed in sections \ref{subsec:first order
  homogeneous}--\ref{subsec:separable} are all special cases of the
strategy in section \ref{subsec:exact}. Hence, all first-order linear
differential equations that can be solved with the strategies
discussed in section \ref{sec:first order} may be solved according to
section \ref{subsec:exact}. However, to minimize work, the following
strategy is advisable:
\begin{enumerate}
\item If the equation is linear and homogeneous, use the strategy of
  section \ref{subsec:first order homogeneous}.
\item If the equation is linear and nonhomogeneous, use the strategy
  of section \ref{subsec:first order nonhomogeneous}.
\item If the equation is separable, use the strategy of section
  \ref{subsec:separable}.
\item Otherwise, use the strategy of section \ref{subsec:exact}.
\end{enumerate}

\section{Second-Order Differential Equations}
\label{sec:second order}

\subsection{Characterization of Second-Order Equations}
\label{subsec:second order characterization}

A \term{second-order differential equation} is an equation of the form
\begin{equation*}
  \deriv[2]{y}{t} = f\left(t, y, \deriv{y}{t}\right).
\end{equation*}
If two additional equations of the form
\begin{align*}
  y (t_0) &= y _0 \\
  y'(t_0) &= y'_0
\end{align*}
are also given, then the three equations together are called an
\term{initial-value problem}. Every initial-value problem has a unique
solution; however, if initial conditions are not provided, the general
solution will have two arbitrary constants.\footnote{This requires
  that $f$ be sufficiently well-behaved.} A second-order differential
equation of the form
\begin{equation} \label{eq:second order nonhomogeneous}
  \deriv[2]{y}{t} + p(t)\deriv{y}{t} + q(t)y = g(t)
\end{equation}
is called \term{linear}. If $g(t) = 0$ then the equation is of the
form
\begin{equation} \label{eq:second order homogeneous}
  \deriv[2]{y}{t} + p(t)\deriv{y}{t} + q(t)y = 0
\end{equation}
and it is called \term{homogeneous}; otherwise, it is called
\term{nonhomogeneous}.

Two functions $y_1(t)$ and $y_2(t)$ are called \term{linearly
  independent} if neither is a scalar multiple of the
other.\footnote{As the scalar multiple may be $0$, this means that
  neither function may be the zero function.} The \term{Wronskian} of
two functions $y_1(t)$ and $y_2(t)$ is the quantity
\begin{equation*}
  \W{y_1, y_2}(t) = \dmat{y_1(t) & y_2(t) \\ y'_1(t) & y'_2(t)}.
\end{equation*}
The Wronskian of two linearly independent solutions of \eq{second
  order homogeneous} is nonzero for all $t$, while the Wronskian of
two linearly dependent solutions of \eq{second order homogeneous} is
zero for all $t$.

If $y_1(t)$ and $y_2(t)$ are two linearly independent solutions of
\eq{second order homogeneous}, then the general solution to \eq{second
  order homogeneous} is
\begin{equation*}
  y(t) = C_1y_1(t) + C_2y_2(t).
  \footnote{This requires that $p$ and $q$ be sufficiently
    well-behaved.}
\end{equation*}
If $y_1(t)$ and $y_2(t)$ are two linearly independent solutions of
\eq{second order homogeneous} and $\psi(t)$ is a solution of
\eq{second order nonhomogeneous}, then the general solution to
\eq{second order nonhomogeneous} is
\begin{equation*}
  y(t) = C_1y_1(t) + C_2y_2(t) + \psi(t).
  \footnote{This requires that $p$, $q$, and $g$ be sufficiently
    well-behaved.}
\end{equation*}

\subsection{Direct Reduction of Order}
\label{subsec:direct reduction}

If $y$ does not appear in a second-order differential equation, i.e.
it is of the form
\begin{equation*}
  \deriv[2]{y}{t} = f\left(t, \deriv{y}{t}\right),
\end{equation*}
then letting $v = \fderiv{y}{t}$ converts it to a first-order
equation:
\begin{equation*}
  \deriv{v}{t} = f(t, v).
\end{equation*}
One constant of integration is introduced in the solution of this
first-order equation, and a second is introduced in recovering $y$:
\begin{equation*}
  y(t) = \int v(t) \,dt.
\end{equation*}
In the context of an initial-value problem, the initial condition
$y'(t_0) = y'_0$ is used in the solution of the first-order equation
and the initial condition $y(t_0) = y_0$ is used in recovering $y$.

\subsection{Reduction of Order}
\label{subsec:reduction of order}

Suppose that $y_1(t)$ is a nonzero solution of the homogeneous
second-order linear differential equation
\begin{equation} \label{eq:second order homogeneous reduction of order}
  \deriv[2]{y}{t} + p(t)\deriv{y}{t} + q(t)y = 0,
\end{equation}
and suppose that we would like to find a second, linearly independent,
solution $y_2(t)$. Since both $y_1(t)$ and $y_2(t)$ are nonzero, it
follows that
\begin{equation*}
  y_2(t) = y_1(t)v(t)
\end{equation*}
for some function $v(t)$. The first and second derivatives of $y_2(t)$
follow from the product rule:
\begin{align*}
  \deriv   {y_2}{t} &= y_1' (t)v(t) +  y_1 (t)v'(t) \\
  \deriv[2]{y_2}{t} &= y_1''(t)v(t) +  y_1'(t)v'(t) + y_1'(t)v' (t)
                                                    + y_1 (t)v''(t) \\
                    &= y_1''(t)v(t) + 2y_1'(t)v'(t) + y_1 (t)v''(t).
\end{align*}
Plugging these into \eq{second order homogeneous reduction of order},
we find that:
\begin{align*}
     y_1''(t)v(t) + 2y_1'(t)v'(t) + y_1(t)v''(t)
   + p(t)y_1'(t)v(t) + p(t)y_1(t)v'(t) + q(t)y_1(t)v(t)
  &= 0 \\
     y_1(t)v''(t) + \Big[2y_1'(t) + p(t)y_1(t)\Big]v'(t)
   + \Big[y_1''(t) + p(t)y_1'(t) + q(t)y_1(t)\Big]v(t)
  &= 0.
\end{align*}
Now, since $y_1(t)$ is a solution of \eq{second order homogeneous
  reduction of order}, we know that $y_1''(t) + p(t)y_1'(t) +
q(t)y_1(t) = 0$, and hence
\begin{equation*}
  y_1(t)v''(t) + \Big[2y_1'(t) + p(t)y_1(t)\Big]v'(t) = 0.
\end{equation*}
To put the equation in standard form, we divide through by $y_1(t)$,
which is nonzero:
\begin{equation*}
  v''(t) + \left[2\left(\frac{y_1'(t)}{y_1(t)}\right)
         + p(t)\right]v'(t) = 0.
\end{equation*}
Using the strategy of section \ref{subsec:direct reduction}, we let
$w(t) = v'(t)$ and obtain:
\begin{equation*}
  w'(t) + \left[2\left(\frac{y_1'(t)}{y_1(t)}\right)
        + p(t)\right]w(t) = 0.
\end{equation*}
This is a homogeneous first-order linear differential equation, and
its solution\footnote{We only need one, nonzero, solution, so we set
  $C = 1$.} via the techniques of section \ref{subsec:first order
  homogeneous} is:
\begin{align*}
  w(t) &= \exp\left(-\int 2\left(\frac{y_1'(t)}{y_1(t)}\right)
         + p(t) \,dt\right) \\
       &= \frac{\exp\left(-\int p(t) \,dt\right)}
               {\exp\left(2 \int \frac{y_1'(t)}{y_1(t)} \,dt\right)} \\
       &= \frac{\exp\left(-\int p(t) \,dt\right)}
               {\Big[\exp\big(\ln|y_1(t)|\big)\Big]^2} \\
       &= \frac{\exp\left(-\int p(t) \,dt\right)}{y_1^2(t)}.
\end{align*}
Hence,
\begin{equation*}
  v(t) = \int w(t) \,dt = \int \frac{\exp\left(-\int p(t) \,dt\right)}
                                    {y_1^2(t)} \,dt,
\end{equation*}
where we have set $C = 0$ since we need only one solution, and
\begin{equation*}
    y_2(t)
  = y_1(t)v(t)
  = y_1(t) \int \frac{\exp\left(-\int p(t) \,dt\right)}
                     {y_1^2(t)}
                \,dt.
\end{equation*}
is a second solution of \eq{second order homogeneous reduction of
  order}. Since $v(t)$ is obtained by integrating a nonzero function,
it cannot be a constant function; it then follows that $y_1(t)$ and
$y_2(t)$ are linearly independent.

\subsection{Homogeneous Linear Equations with Constant Coefficients}
\label{subsec:constant coefficients}

Consider now the general homogeneous second-order linear differential
equation with constant coefficients,
\begin{equation} \label{eq:constant coefficients}
  a \deriv[2]{y}{t} + b \deriv{y}{t} + cy = 0.
\end{equation}
The trick to solving this equation is to guess that solutions may be
of the form $y(t) = e^{rt}$. In that case, we have $y'(t) = re^{rt}$
and $y''(t) = r^2e^{rt}$; plugging these into \eq{constant
  coefficients} gives
\begin{equation*}
  (ar^2 + br + c)e^{rt} = 0.
\end{equation*}
Since $e^{rt}$ is never zero, we can divide through by it, leaving us
with a quadratic equation for $r$:
\begin{equation*}
  ar^2 + br + c = 0.
\end{equation*}
The quadratic formula then yields
\begin{equation*}
  r = \frac{-b \pm \sqrt{b^2 - 4ac}}{2a}.
\end{equation*}
Depending on the value of $b^2 - 4ac$, we may obtain two real roots,
two complex roots, or one real root.

\subsubsection{Two Real Roots
  \texorpdfstring{$(b^2 - 4ac > 0)$}{(b^2 - 4ac > 0)}}
\label{subsubsec:constant coefficients real roots}

In this case, we have the two linearly independent solutions
\begin{align*}
  y_1(t) &= e^{r_1t} \\
  y_2(t) &= e^{r_2t},
\end{align*}
where
\begin{align*}
  r_1 &= \frac{-b + \sqrt{b^2 - 4ac}}{2a} \\
  r_2 &= \frac{-b - \sqrt{b^2 - 4ac}}{2a}.
\end{align*}
The general solution is
\begin{equation*}
  y(t) = C_1e^{r_1t} + C_2e^{r_2t}.
\end{equation*}

\subsubsection{Two Complex Roots
  \texorpdfstring{$(b^2 - 4ac < 0)$}{(b^2 - 4ac < 0)}}
\label{subsubsec:constant coefficients complex roots}

In this case, we have the two linearly independent solutions
\begin{align*}
  z_1(t) &= e^{(\lambda + i\mu)t} \\
  z_2(t) &= e^{(\lambda - i\mu)t},
\end{align*}
where
\begin{align*}
  \lambda &= -\frac{b}{2a}, \\
  \mu &= \frac{\sqrt{4ac - b^2}}{2a}.
\end{align*}
However, these solutions are complex-valued and we only want real
solutions. To remedy this problem, suppose that $z(t) = u(t) + iv(t)$
is a complex-valued solution of \eq{constant coefficients}, which is
reprinted here:
\begin{equation*}
  a \deriv[2]{y}{t} + b \deriv{y}{t} + cy = 0.
\end{equation*}
Then we would have:
\begin{align*}
    a \deriv[2]{u}{t} + ia \deriv[2]{v}{t}
  + b \deriv   {u}{t} + ib \deriv   {v}{t}
  + cu + icv &= 0 \\
     \left[a \deriv[2]{u}{t} + b \deriv{u}{t} + cu\right]
  + i\left[a \deriv[2]{v}{t} + b \deriv{v}{t} + cv\right] &= 0.
\end{align*}
For the complex-valued left-hand side to equal $0$, both its real and
imaginary parts would have to equal zero. This means that both $u$ and
$v$ are real-valued solutions of \eq{constant coefficients}.

In other words, if $z(t)$ is a complex-valued solution of \eq{constant
  coefficients}, then both $\Re{z(t)}$ and $\Im{z(t)}$ are real-valued
solutions of \eq{constant coefficients}. Since, by Euler's formula,
\begin{equation*}
  z_1(t) = e^{(\lambda + i\mu)t}
         = e^{\lambda t + i\mu t}
         = e^{\lambda t}\big[\cos(\mu t) + i \sin(\mu t)\big],
\end{equation*}
we have
\begin{align*}
  y_1(t) &= e^{\lambda t} \cos(\mu t) \\
  y_2(t) &= e^{\lambda t} \sin(\mu t)
\end{align*}
as linearly independent real-valued solutions of \eq{constant
  coefficients}. According to the discussion of section
\ref{subsec:second order characterization}, this is sufficient to
characterize the general solution of \eq{constant coefficients}.
However, it is natural to wonder why we are allowed to discard
$z_2(t)$: might it not provide additional solutions? This turns out
not to be the case: we have
\begin{equation*}
  z_2(t) = e^{(\lambda - i\mu)t}
         = e^{\lambda t - i\mu t}
         = e^{\lambda t} \big[\cos(\mu t) - i \sin(\mu t)\big],
\end{equation*}
and thus
\begin{align*}
  \Re{z_2(t)} &=  y_1(t) \\
  \Im{z_2(t)} &= -y_2(t).
\end{align*}
Thus, the components of $z_2(t)$ are just scalar multiples of the
components of $z_1(t)$, and are already accounted for in the general
solution,
\begin{equation*}
  y(t) = e^{\lambda t} \big[C_1 \cos(\mu t) + C_2 \sin(\mu t)\big].
\end{equation*}

\subsubsection{One Real Root
  \texorpdfstring{$(b^2 - 4ac = 0)$}{(b^2 - 4ac = 0)}}
\label{subsubsec:constant coefficients equal roots}

In this case, we have one solution $y_1(t) = e^{r_1t}$, where $r_1 =
-b/(2a)$, and can use two different methods to find a second.

The first method is reduction of order. To use the results of section
\ref{subsec:reduction of order}, we must first divide through by $a$
in order to put \eq{constant coefficients} in standard form:
\begin{equation*}
    \deriv[2]{y}{t}
  + \left(\frac{b}{a}\right) \deriv{y}{t}
  + \left(\vphantom{\frac{b}{a}}\frac{c}{a}\right)y
  = 0.
\end{equation*}
We now have $p(t) = b/a$, so from our work in section
\ref{subsec:reduction of order} we have:
\begin{align*}
  y_2(t) &= y_1(t) \int \frac{\exp\left(-\int p(t) \,dt\right)}
                             {y_1^2(t)} \,dt \\
         &= e^{r_1t} \int \frac{\exp\left(-\int b/a \,dt\right)}
                               {\big[e^{-b/(2a)}\big]^2} \,dt \\
         &= e^{r_1t} \int \frac{e^{-b/a}}{e^{-b/a}} \,dt \\
         &= e^{r_1t} \int dt \\
         &= te^{r_1t}.
\end{align*}
We now have two linearly independent solutions $y_1(t)$ and $y_2(t)$.

To use the second method, we can define the operator $L$ as
\begin{equation} \label{eq:constant coefficients operator}
  L\big[y\big] = a \deriv[2]{y}{t} + b \deriv{y}{t} + cy.
\end{equation}
Plugging $y = e^{rt}$ into \eq{constant coefficients operator} gives
us
\begin{equation*}
  L\big[e^{rt}\big] = (ar^2 + br + c)e^{rt},
\end{equation*}
and since $r = r_1$ is a root of the quadratic equation $ar^2 + br +
c$, it follows that
\begin{equation*}
  L\big[e^{r_1t}\big] = 0,
\end{equation*}
which shows that $y_1(t) = e^{r_1t}$ is a solution of \eq{constant
  coefficients}. However, we already knew this. To find a second
solution, first notice that we can use the equality of mixed partials
to conclude:
\begin{align*}
     \pderiv{}{r} L\big[y\big]
  &= \pderiv{}{r} \left[  a \deriv[2]{y}{t}
                        + b \pderiv{y}{t}
                        + cy\right] \\
  &= a \ppderiv{y}{t}{2}{r}{}
   + b \ppderiv{y}{t}{}{r}{}
   + c \pderiv{y}{r} \\
  &= a \ppderiv{y}{r}{}{t}{2}
   + b \ppderiv{y}{r}{}{t}{}
   + c \pderiv{y}{r} \\
  &= a \pderiv[2]{}{t} \left[\pderiv{y}{r}\right]
   + b \pderiv{}{t} \left[\pderiv{y}{r}\right]
   + c \pderiv{y}{r} \\
  &= L\left[\pderiv{y}{r}\right].
\end{align*}
Now observe that because $r = r_1$ is a double root of $ar^2 + br +
c$, we can factor the latter expression to $a(r - r_1)^2$. Thus,
\begin{equation*}
  L\big[e^{rt}\big] = a(r - r_1)^2e^{rt}.
\end{equation*}
Differentiating both sides of this relation with respect to $r$ gives:
\begin{align*}
     L\left[\pderiv{}{r} e^{rt}\right]
  &= a(r - r_1)^2 \pderiv{}{r} \Big[e^{rt}\Big]
   + ae^{rt} \pderiv{}{r} \Big[(r - r_1)^2\Big] \\
  L\big[te^{rt}\big] &= at(r - r_1)^2e^{rt} + 2a(r - r_1)e^{rt}.
\end{align*}
Finally, by letting $r = r_1$, we see that
\begin{equation*}
  L\big[te^{r_1t}\big] = 0,
\end{equation*}
which means that $y_2(t) = te^{r_1t}$ is a second solution of
\eq{constant coefficients}. To be honest, there is no reason in
particular why one might be inspired to differentiate the general form
of a solution with respect to $r$. However, it turns out that this
technique works in many other cases as well, and it is often much,
much easier than the method of reduction of order.

Either way, the general solution is
\begin{equation*}
  y(t) = (C_1 + C_2t)e^{r_1t}.
\end{equation*}

\subsection{Variation of Parameters}
\label{subsec:variation of parameters}

Consider the nonhomogeneous second-order linear differential equation
\begin{equation} \label{eq:variation of parameters nonhomogeneous}
  \deriv[2]{y}{t} + p(t) \deriv{y}{t} + q(t)y = g(t)
\end{equation}
and the corresponding homogeneous equation
\begin{equation} \label{eq:variation of parameters homogeneous}
  \deriv[2]{y}{t} + p(t) \deriv{y}{t} + q(t)y = 0.
\end{equation}

Suppose that we have two linearly independent solutions $y_1(t)$ and
$y_2(t)$ of \eq{variation of parameters homogeneous}. Then the general
solution of \eq{variation of parameters homogeneous} is
\begin{equation*}
  y(t) = C_1y_1(t) + C_2y_2(t).
\end{equation*}
Suppose that we now wish to find a particular solution $\psi(t)$ to
\eq{variation of parameters nonhomogeneous}. We may do this by
supposing such a solution is of the form
\begin{equation}
  \psi(t) = u_1(t)y_1(t) + u_2(t)y_2(t);
\end{equation}
that is, by letting the parameters $C_1$ and $C_2$ vary with time.
Notice that we have replaced the problem of finding an appropriate
single function $\psi(t)$ by the problem of finding two appropriate
functions $u_1(t)$ and $u_2(t)$; in a sense, we have added an
additional degree of freedom to our problem. As such, we are justified
in adding an additional constraint on $u_1(t)$ and $u_2(t)$ without
making it so that the solution set of our problem is empty.

We now compute
\begin{equation*}
    \deriv{\psi}{t}
  = u_1'(t)y_1(t) + u_1(t)y_1'(t) + u_2'(t)y_2(t) + u_2(t)y_2'(t).
\end{equation*}
Note that if
\begin{equation} \label{eq:variation of parameters condition}
  u_1'(t)y_1(t) + u_2'(t)y_2(t) = 0,
\end{equation}
then the second derivative of $\psi(t)$ will have no second-order
derivatives of $u_1(t)$ or $u_2(t)$. In accordance with our discussion
above, we arbitrarily assume that \eq{variation of parameters
  condition} holds. In this case,
\begin{equation*}
  \deriv{\psi}{t} = u_1(t)y_1'(t) + u_2(t)y_2'(t),
\end{equation*}
and
\begin{equation*}
  \deriv[2]{\psi}{t} = u_1'(t)y_1'(t)
                     + u_1(t)y_1''(t)
                     + u_2'(t)y_2'(t)
                     + u_2(t)y_2''(t).
\end{equation*}
Plugging these expressions into \eq{variation of parameters
  nonhomogeneous} gives
\begin{align*}
  &\phantom{{}+{}} u_1'(t)y_1'(t)
              +    u_1(t)y_1''(t)
              +    u_2'(t)y_2'(t)
              +    u_2(t)y_2''(t) \\
  &+ p(t)u_1(t)y_1'(t)
   + p(t)u_2(t)y_2'(t)
   + q(t)u_1(t)y_1(t)
   + q(t)u_2(t)y_2(t)
   = g(t),
\end{align*}
which may be factored to
\begin{align*}
  &\phantom{{}+{}} u_1(t)\Big[y_1''(t)
              +    p(t)y_1'(t)
              +    q(t)y_1(t)\Big]
              +    u_2(t)\Big[y_2''(t)
              +    p(t)y_2'(t)
              +    q(t)y_2(t)\Big] \\
  &           +    u_1'(t)y_1'(t)
              +    u_2'(t)y_2'(t)
              =    g(t).
\end{align*}
Because $y_1(t)$ and $y_2(t)$ are solutions of \eq{variation of
  parameters homogeneous}, the first two terms cancel and we are left
with
\begin{equation*}
  u_1'(t)y_1'(t) + u_2'(t)y_2'(t) = g(t).
\end{equation*}
Surprisingly, when we combine this relation with \eq{variation of
  parameters condition}, we find that we can solve algebraically for
$u_1'(t)$ and $u_2'(t)$. Doing so gives:
\begin{align*}
  u_1'(t) &= \frac{-g(t)y_2(t)}{y_1(t)y_2'(t) - y_1'(t)y_2(t)}, \\
  u_2'(t) &= \frac{ g(t)y_1(t)}{y_1(t)y_2'(t) - y_1'(t)y_2(t)}.
\end{align*}
Hence, a particular solution of \eq{variation of parameters
  nonhomogeneous} is given by:
\begin{equation*}
  \psi(t) = y_1(t) \int \frac{-g(t)y_2(t)}{y_1(t)y_2'(t)
          - y_1'(t)y_2(t)} \,dt
          + y_2(t) \int \frac{g(t)y_1(t)}{y_1(t)y_2'(t)
          - y_1'(t)y_2(t)} \,dt.
\end{equation*}
The method of variation of parameters will solve any nonhomogeneous
second-order linear differential equation, assuming that the general
solution to the corresponding homogeneous equation has already been
computed, but it is often impractical because of the difficulty of the
integrals it requires.

\subsection{Method of Judicious Guessing}
\footnote{This is also known as the method of undetermined
  coefficients.}
\label{subsec:judicious guessing}

In this section we will develop a method for solving any
nonhomogeneous second-order linear differential equation of the form
\begin{equation*}
    a \deriv[2]{y}{t} + b \deriv{y}{t} + cy
  = \sum_{i=1}^N p_i(t)e^{\lambda_i t}q_i(t),
\end{equation*}
where each of the $p_i(t)$ is a polynomial in $t$ and each of the
$q_i(t)$ is either $\cos(\mu_i t)$ or $\sin(\mu_i t)$.

\subsubsection{Polynomial}
\label{subsubsec:judicious guessing polynomial}

We will start with the simple case where the right-hand side is just a
polynomial:
\begin{equation} \label{eq:judicious guessing polynomial}
    a \deriv[2]{y}{t} + b \deriv{y}{t} + cy
  = a_0 + a_1t + a_2t^2 + \cdots + a_nt^n.
\end{equation}
Notice that if $y$ is a polynomial of degree $n$, then the left-hand
side will also be a polynomial of degree $n$. Thus it is reasonable to
suppose that
\begin{equation} \label{eq:judicious guessing solution form}
  y(t) = A_0 + A_1t + A_2t^2 + \cdots
       + A_{n-2}t^{n-2} + A_{n-1}t^{n-1} + A_nt^n.
\end{equation}
Then,
\begin{align*}
  y' (t) &= A_1 + 2A_2t + 3A_3t^2 + \cdots
          + (n-1)A_{n-1}t^{n-2} + nA_nt^{n-1} \\
  y''(t) &= 2A_2 + 6A_3t + 12A_4t^2 + \cdots
          + n(n-1)A_nt^{n-2},
\end{align*}
and plugging these into \eq{judicious guessing polynomial} gives
\begin{align*}
  &\phantom{{}+{}} 2aA_2 + 6aA_3t + 12aA_4t^2 + \cdots
              +    n(n-1)aA_nt^{n-2} \\
  &           +    bA_1 + 2bA_2t + 3bA_3t^2 + \cdots
              +    (n-1)bA_{n-1}t^{n-2} + nbA_nt^{n-1} \\
  &           +    cA_0 + cA_1t + cA_2t^2 + \cdots
              +    cA_{n-2}t^{n-2} + cA_{n-1}t^{n-1} + cA_nt^n \\
  &           =    a_0 + a_1t + a_2t^2 + \cdots + a_nt^n.
\end{align*}
We can group like terms on the left-hand side to obtain:
\begin{align*}
  &\phantom{{}+{}} \big[cA_0 + bA_1 + 2aA_2\big]
              +    \big[cA_1 + 2bA_2 + 6aA_3\big]t
              +    \big[cA_2 + 3bA_3 + 12aA_4\big]t^2 + \cdots \\
  &           +    \big[  cA_{n-2}
                        + (n-1)bA_{n-1}
                        + n(n-1)aA_n\big]t^{n-2}
              +    \big[cA_{n-1} + nbA_n\big]t^{n-1}
              +    cA_nt^n \\
  &           =    a_0 + a_1t + a_2t^2 + \cdots + a_nt^n.
\end{align*}
Setting equivalent powers of $t$ equal and solving for the
coefficients $A_0, A_1, \ldots, A_n$, we find that:
\begin{align*}
  A_n &= \frac{a_n}{c} \\
  A_{n-1} &= \frac{a_{n-1} - nbA_n}{c} \\
  A_{n-2} &= \frac{a_{n-2} - (n-1)bA_{n-1} - n(n-1)aA_n}{c} \\
      &\cvdots \\
  A_2 &= \frac{a_2 - 3bA_3 - 12aA_4}{c} \\
  A_1 &= \frac{a_1 - 2bA_2 - 6aA_3}{c} \\
  A_0 &= \frac{a_0 - bA_1 - 2aA_2}{c}.
\end{align*}
If $c \neq 0$, then we have a particular solution of \eq{judicious
  guessing polynomial} given by \eq{judicious guessing solution form}.
On the other hand, if $c = 0$, we may apply the reasoning of section
\ref{subsec:direct reduction} and let $v = \fderiv{y}{t}$. On doing
so, \eq{judicious guessing polynomial} becomes
\begin{equation} \label{eq:judicious guessing polynomial v}
  a \deriv{v}{t} + bv = a_0 + a_1t + a_2t^2 + \cdots + a_nt^n.
\end{equation}
As before, we assume that
\begin{equation} \label{eq:judicious guessing solution form v}
  v(t) = A_0 + A_1t + \cdots + A_nt^n.
\end{equation}
It would be easy to solve for the coefficients as we did before, but
to simplify our work note that \eq{judicious guessing polynomial v}
can be obtained from \eq{judicious guessing polynomial} by making the
replacements
\begin{align*}
  y &\to v \\
  a &\to 0 \\
  b &\to a \\
  c &\to b.
\end{align*}
Thus we can re-use our work from earlier:
\begin{align*}
  A_n &= \frac{a_n}{b} \\
  A_{n-1} &= \frac{a_{n-1} - naA_n}{b} \\
  A_{n-2} &= \frac{a_{n-2} - (n-1)aA_{n-1}}{b} \\
      &\cvdots \\
  A_2 &= \frac{a_2 - 3aA_3}{b} \\
  A_1 &= \frac{a_1 - 2aA_2}{b} \\
  A_0 &= \frac{a_0 - aA_1}{b}.
\end{align*}
This gives us the coefficients for $v(t)$; to obtain $y(t)$ we simply
integrate \eq{judicious guessing solution form v}:
\begin{equation*}
  y(t) = A_0t + \frac{A_1t^2}{2} + \frac{A_2t^3}{3}
       + \cdots + \frac{a_nt^{n+1}}{n+1}.
\end{equation*}
We have set $C = 0$ because we only need one particular solution. (Any
constant would be a solution of the homogeneous equation corresponding
to \eq{judicious guessing polynomial} if $c = 0$.) Of course, here we
run into trouble if $b = 0$. Now, we could re-use our work above, but
note that if $b$ and $c$ are both zero, then solving \eq{judicious
  guessing polynomial} is trivial: we simply integrate twice and
divide by $a$. This gives
\begin{equation*}
  y(t) = \frac{1}{a} \left[\frac{a_0t^2}{2}
       + \frac{a_1t^3}{6}
       + \frac{a_2t^4}{12}
       + \cdots
       + \frac{a_nt^{n+2}}{(n+1)(n+2)}\right].
\end{equation*}
Looking at the results of this section, we may conclude that
particular solutions of \eq{judicious guessing polynomial} occur in
the forms:
\begin{align*}
  A_0 + A_1t + \cdots + A_nt^n,
  && \text{if $c \neq 0$}, \\
  t\Big[A_0 + A_1t + \cdots + A_nt^n\Big],
  && \text{if $c = 0$ but $b \neq 0$}, \\
  t^2\Big[A_0 + A_1t + \cdots + A_nt^n\Big],
  && \text{if $c = b = 0$}.
\end{align*}

\subsubsection{Polynomial with Exponential}
\label{subsubsec:judicious guessing exponential}

We now consider equations of the form
\begin{equation} \label{eq:judicious guessing exponential}
    a \deriv[2]{y}{t} + b \deriv{y}{t} + cy
  = (a_0 + a_1t + a_2t^2 + \cdots + a_nt^n)e^{\gamma t}.
\end{equation}
To reduce \eq{judicious guessing exponential} to \eq{judicious
  guessing polynomial}, we assume that $y(t) = v(t) e^{\gamma t}$.
From this, we have
\begin{align*}
     \deriv{y}{t}
  &= \left[\deriv{v}{t} + \gamma v\right] e^{\gamma t} \\
     \deriv[2]{y}{t}
  &= \left[\deriv[2]{v}{t} + 2\gamma \deriv{v}{t}
                           + \gamma^2 v\right] e^{\gamma t},
\end{align*}
and from \eq{judicious guessing exponential} it follows that
\begin{align*}
     \left[a \deriv[2]{v}{t}
   + 2a\gamma \deriv{v}{t}
   + a\gamma^2 v
   + b \deriv{v}{t}
   + b\gamma v
   + cv\right] e^{\gamma t}
  &= (a_0 + a_1t + a_2t^2 + \cdots + a_nt^n)e^{\gamma t} \\
     a \deriv[2]{v}{t}
   + \big[2a\gamma
   + b\big] \deriv{v}{t}
   + \big[a\gamma^2
   + b\gamma
   + c\big] v
  &= a_0 + a_1t + a_2t^2 + \cdots + a_nt^n.
\end{align*}
This equation can now be solved for $v(t)$ according to the techniques
of the previous section; then, $y(t) = v(t)e^{\gamma t}$. From the
summary given in the previous section, we may conclude that particular
solutions of \eq{judicious guessing exponential} occur in the forms:
\begin{align*}
        \Big[A_0 + A_1t + \cdots + A_nt^n\Big] e^{\gamma t},
  && \text{if $a\gamma^2 + b\gamma + c \neq 0$}, \\
     t  \Big[A_0 + A_1t + \cdots + A_nt^n\Big] e^{\gamma t},
  && \text{if $a\gamma^2 + b\gamma + c = 0$
           but $2a\gamma + b \neq 0$}, \\
     t^2\Big[A_0 + A_1t + \cdots + A_nt^n\Big] e^{\gamma t},
  && \text{if $a\gamma^2 + b\gamma + c = 2a\gamma + b = 0$}.
\end{align*}
However, notice that the expressions appearing above have an
interesting relationship to the homogeneous equation corresponding to
\eq{judicious guessing exponential},
\begin{equation} \label{eq:judicious guessing homogeneous}
  a \deriv[2]{y}{t} + b \deriv{y}{t} + cy = 0.
\end{equation}
In particular, as we saw in section \ref{subsec:constant
  coefficients}, if $a\gamma^2 + b\gamma + c = 0$, then $y = e^{\gamma
  t}$ is a solution of \eq{judicious guessing homogeneous}. If we also
have $2a\gamma + b = 0$, or equivalently $\gamma = -b/(2a)$, then
$\gamma$ is a double root of $a\gamma^2 + b\gamma + c = 0$. In this
case, both $y = e^{\gamma t}$ and $y = te^{\gamma t}$ are solutions of
\eq{judicious guessing homogeneous}. So we may conclude that
particular solutions of \eq{judicious guessing exponential} occur in
the forms:
\begin{align*}
        \Big[A_0 + A_1t + \cdots + A_nt^n\Big] e^{\gamma t},
  && \text{if $y = e^{\gamma t}$ does not solve
     \eq{judicious guessing homogeneous}}, \\
     t  \Big[A_0 + A_1t + \cdots + A_nt^n\Big] e^{\gamma t},
  && \text{if $y = e^{\gamma t}$ solves
           \eq{judicious guessing homogeneous}
           but $y = te^{\gamma t}$ does not}, \\
     t^2\Big[A_0 + A_1t + \cdots + A_nt^n\Big] e^{\gamma t},
  && \text{if both $y = e^{\gamma t}$ and $y = te^{\gamma t}$
           solve \eq{judicious guessing homogeneous}}.
\end{align*}

\subsubsection{Polynomial with Trigonometric Function}
\label{subsubsec:judicious guessing trigonometric}

We now consider equations of the form
\begin{equation} \label{eq:judicious guessing trigonometric}
    a \deriv[2]{y}{t} + b \deriv{y}{t} + c \deriv{y}{t}
  = (a_0 + a_1t + a_2t^2 + \cdots + a_nt^n) e^{\lambda t} \cos(\mu t).
\end{equation}
To solve this equation, note that
\begin{equation*}
    \Re{(a_0 + a_1t + a_2t^2 + \cdots + a_nt^n)
        e^{(\lambda + i \mu) t}}
  = (a_0 + a_1t + a_2t^2 + \cdots + a_nt^n) e^{\lambda t} \cos(\mu t).
\end{equation*}
According to our discussion in section \ref{subsubsec:constant
  coefficients complex roots}, if we can find a complex-valued
solution $z(t)$ of
\begin{equation*}
    a \deriv[2]{y}{t} + b \deriv{y}{t} + c \deriv{y}{t}
  = (a_0 + a_1t + a_2t^2 + \cdots + a_nt^n) e^{(\lambda + i \mu) t},
\end{equation*}
then $\Re{z(t)}$ will be a real-valued solution of \eq{judicious
  guessing trigonometric}. If we let $\gamma = \lambda + i \mu$, then
the above equation may be solved using the techniques of the previous
section, although the coefficients $A_0, A_1, \ldots, A_n$ will now be
complex-valued. Taking the real part of the resulting function gives a
particular solution to \eq{judicious guessing trigonometric}.

Tracing the calculations of this method allows us to conclude that
particular solutions of \eq{judicious guessing trigonometric} occur in
the forms:
\begin{align*}
       \Big[A_0 + A_1t + \cdots + A_nt^n\Big] e^{\lambda t} \cos(\mu t)
  +    \Big[B_0 + B_1t + \cdots + B_nt^n\Big] e^{\lambda t} \sin(\mu t),
    \qquad \qquad \\
  \text{if $y = e^{\lambda t} \cos(\mu t)$ does not
        solve \eq{judicious guessing homogeneous}}, \\
    t  \Big[A_0 + A_1t + \cdots + A_nt^n\Big] e^{\lambda t} \cos(\mu t)
  + t  \Big[B_0 + B_1t + \cdots + B_nt^n\Big] e^{\lambda t} \sin(\mu t),
    \qquad \qquad \\
  \text{if $y = e^{\lambda t} \cos(\mu t)$ solves
        \eq{judicious guessing homogeneous} but
        $y = te^{\lambda t} \cos(\mu t)$ does not}, \\
    t^2\Big[A_0 + A_1t + \cdots + A_nt^n\Big] e^{\lambda t} \cos(\mu t)
  + t^2\Big[B_0 + B_1t + \cdots + B_nt^n\Big] e^{\lambda t} \sin(\mu t),
    \qquad \qquad \\
  \text{if both $y = e^{\lambda t} \cos(\mu t)$ and
        $y = te^{\lambda t} \cos(\mu t)$ solve
        \eq{judicious guessing homogeneous}}.
\end{align*}
Of course, by replacing $\cos$ with $\sin$ and $\OpRe$ with $\OpIm$,
we may instead obtain solutions of equations of the form
\begin{equation*}
    a \deriv[2]{y}{t} + b \deriv{y}{t} + c \deriv{y}{t}
  = (a_0 + a_1t + a_2t^2 + \cdots + a_nt^n) e^{\lambda t} \sin(\mu t).
\end{equation*}

\subsubsection{General Equation}
\label{subsubsec:judicious guessing general}

Observe that our work in section \ref{subsubsec:judicious guessing
  trigonometric} allows us to solve any nonhomogeneous second-order
linear differential equation of the form
\begin{equation} \label{eq:judicious guessing general i}
    a \deriv[2]{y}{t} + b \deriv{y}{t} + cy
  = p_i(t)e^{\lambda_i t}q_i(t),
\end{equation}
where $p_i(t)$ is a polynomial in $t$ and $q_i(t)$ is either
$\cos(\mu_i t)$ or $\sin(\mu_i t)$. (We can recover the equations of
sections \ref{subsubsec:judicious guessing exponential} and
\ref{subsubsec:judicious guessing polynomial} by setting $\mu = 0$ and
$\lambda = 0$, respectively.) The procedure for solving the general
equation
\begin{equation} \label{eq:judicious guessing general}
    a \deriv[2]{y}{t} + b \deriv{y}{t} + cy
  = \sum_{i=1}^N p_i(t)e^{\lambda_i t}q_i(t)
\end{equation}
presented at the beginning of section \ref{subsec:judicious guessing}
is simple, if tedious for large $N$. For each $i$ from $1$ to $N$,
find a particular solution $y_i(t)$ to \eq{judicious guessing general
  i}. Then, it is easy to see that a particular solution to
\eq{judicious guessing general} is given by
\begin{equation*}
  y(t) = \sum_{i=1}^N y_i(t).
\end{equation*}

It should also be noted that instead of following the procedures
outlined in sections \ref{subsubsec:judicious guessing
  polynomial}--\ref{subsubsec:judicious guessing general}, we can
instead use the summaries given at the end of each of those sections.
For instance, if the right-hand side of \eq{judicious guessing
  general} is $t^3 e^{2t} \sin(3t) + t^2$, and both $e^{2t} \sin(3t)$
and $te^{2t} \sin(3t)$ are solutions of the corresponding homogeneous
equation, we may assume that $y(t)$ is of the form
\begin{equation*}
  y(t) = t^2\Big[A_0 + A_1t + A_2t^2 + A_3t^3\Big] e^{2t} \cos(3t)
       + t^2\Big[B_0 + B_1t + B_2t^2 + B_3t^3\Big] e^{2t} \sin(3t)
       + C_0 + C_1t + C_2t^2,
\end{equation*}
plug this into \eq{judicious guessing general}, and solve for the
parameters $A_i$, $B_i$, $C_i$ by equating the coefficients of
linearly independent terms. Which of these two approaches is easier
depends on the problem at hand.

\subsection{Power Series Solutions}
\label{subsec:power series}

Consider the homogeneous second-order linear differential equation
\begin{equation} \label{eq:power series homogeneous}
  P(t) \deriv[2]{y}{t} + Q(t) \deriv{y}{t} + R(t) y = 0,
\end{equation}
where $P(t)$, $Q(t)$, and $R(t)$ have series expansions given by:
\begin{align*}
     P(t) = \sum_{n=0}^\infty p_n t^n,
  && Q(t) = \sum_{n=0}^\infty q_n t^n,
  && R(t) = \sum_{n=0}^\infty r_n t^n.
\end{align*}
It is reasonable to suppose that solutions will be of the same form;
that is,
\begin{equation*}
  y(t) = \sum_{n=0}^\infty a_n t^n.
\end{equation*}
Differentiating, we find that
\begin{align*}
  \deriv{y}{t} &= \sum_{n=1}^\infty  n    a_ n    t^{n-1}
                = \sum_{n=0}^\infty (n+1) a_{n+1} t^ n \\
  \deriv[2]{y}{t} &= \sum_{n=2}^\infty  n   (n-1) a_ n    t^{n-2}
                   = \sum_{n=0}^\infty (n+2)(n+1) a_{n+2} t^ n,
\end{align*}
Substituting these series into \eq{power series homogeneous}, we
obtain:
\begin{equation*}
    \sum_{n=0}^\infty p_n t^n \sum_{n=0}^\infty (n+2)(n+1) a_{n+2} t^n
  + \sum_{n=0}^\infty q_n t^n \sum_{n=0}^\infty (n+1)      a_{n+1} t^n
  + \sum_{n=0}^\infty r_n t^n \sum_{n=0}^\infty            a_ n    t^n
  = 0.
\end{equation*}
Now we may use the following elementary property of power series:
\begin{equation*}
    \sum_{n=0}^\infty a_n t^n \sum_{n=0}^\infty b_n t^n
  = \sum_{n=0}^\infty \sum_{k=0}^n a_k b_{n-k} t^n.
\end{equation*}
Doing so yields:
\begin{equation*}
    \sum_{n=0}^\infty \sum_{k=0}^n (k+2)(k+1) p_{n-k} a_{k+2} t^n
  + \sum_{n=0}^\infty \sum_{k=0}^n (k+1) q_{n-k} a_{k+1} t^n
  + \sum_{n=0}^\infty \sum_{k=0}^n r_{n-k} a_k t^n
  = 0.
\end{equation*}
Combining the summations, we find that
\begin{equation*}
  \sum_{n=0}^\infty \left\{\sum_{k=0}^n \bigg[
    (k+2)(k+1) p_{n-k} a_{k+2} + (k+1) q_{n-k} a_{k+1} + r_{n-k} a_k
  \bigg]\right\} t^n = 0,
\end{equation*}
which implies that for every $n \geq 0$ we have
\begin{equation}
  \sum_{k=0}^n \bigg[
    (k+2)(k+1) p_{n-k} a_{k+2} + (k+1) q_{n-k} a_{k+1} + r_{n-k} a_k
  \bigg] = 0.
\end{equation}
For the moment, assume that $p_0 \neq 0$. In that case we can solve
for $a_{n+2}$, as follows. Pulling out the last term of the summation,
for which $k = n$, we obtain
\begin{align*}
  &\phantom{{}+{}} \sum_{k=0}^{n-1} \bigg[
                       (k+2)(k+1) p_{n-k} a_{k+2}
                     + (k+1) q_{n-k} a_{k+1}
                     + r_{n-k} a_k
                   \bigg] \\
  &           + (n+2)(n+1) p_0 a_{n+2}
              +      (n+1) q_0 a_{n+1}
              +            r_0 a_ n
              = 0,
\end{align*}
and solving for $a_{n+2}$ gives
\begin{equation*}
  a_{n+2} = -\left\{
    \frac{\displaystyle \sum_{k=0}^{n-1} \bigg[
              (k+2)(k+1) p_{n-k} a_{k+2}
            + (k+1) q_{n-k} a_{k+1}
            + r_{n-k} a_k
          \bigg] + (n+1) q_0 a_{n+1} + r_0 a_n}
         {\displaystyle \vphantom{\sum_0^0}(n+2)(n+1)p_0}
  \right\}.
\end{equation*}
The above equation holds for $n \geq 0$, and it determines the value
of $a_{n+2}$ in terms of $a_0, a_1, \ldots, a_{n+1}$. Thus, $a_0$ and
$a_1$ are arbitrary, but once their values are specified, the values
of $a_2, a_3, \ldots$ follow from the above equation. Any pair of
values for $a_0$ and $a_1$, then, will determine a solution
\begin{equation*}
  y(t) = \sum_{n=0}^\infty a_n t^n
\end{equation*}
of \eq{power series homogeneous}. Typically, we want two linearly
independent solutions, and typically the easiest way to obtain them is
to use $a_0 = 1$, $a_1 = 0$ for the first solution and $a_0 = 0$, $a_1
= 1$ for the second solution.

Notice from the series for $P(t)$ that $p_0 = P(0)$. Thus, the method
of power series solution will solve any differential equation
\eq{power series homogeneous} for which $P(0) \neq 0$. In particular,
it will solve any equation of the form
\begin{equation} \label{eq:power series frobenius}
  \deriv[2]{y}{t} + p(t) \deriv{y}{t} + q(t) y = 0,
\end{equation}
where $p(t)$ and $q(t)$ may be expressed as power series. Of course,
we may transform \eq{power series homogeneous} into \eq{power series
  frobenius} by dividing through by $P(t)$, but the resulting $p(t) =
Q(t)/P(t)$ and $q(t) = R(t)/P(t)$ will not necessarily be expressible
as power series if $P(0) = 0$. In the next section, we will see how to
solve \eq{power series frobenius} when $p(t)$ and $q(t)$ cannot be
expressed as power series but can still be expressed as more general
series.

We should also note that it is entirely possible to use the above
method with power series about any point (say $t = t_0$), not just $t
= 0$. Simply expand $P(t)$, $Q(t)$, and $R(t)$ as power series about
$t_0$, assume that
\begin{equation*}
  y(t) = \sum_{n=0}^\infty a_n (t - t_0)^n,
\end{equation*}
and follow the same procedure as outlined above. Expanding about a
different point may be useful in order to avoid a singularity: for
instance, suppose that $P(0) = 0$ and we consequently cannot find a
solution $y(t)$ as a power series about $t = 0$. In that case, taking
our power series to be centered at a different point would allow us to
find a solution. Unfortunately, it is likely that the resulting series
for $y(t)$ would not converge near $t = 0$, so to determine the
behavior of $y(t)$ near $t = 0$ we would have to use a different
technique.

Finally, note that if our power series are centered at $t = t_0$ and
we have the initial conditions $y(t_0) = y_0$ and $y'(t_0) = y'_0$,
then it follows easily that $a_0 = y_0$ and $a_1 = y'_0$. On the other
hand, if our initial conditions are given at a different point than
that about which our power series are centered, we will have to solve
for the arbitrary constants as usual.

\subsection{Method of Frobenius}
\label{subsec:frobenius}

As we discussed in section \ref{subsec:power series}, we can solve any
differential equation of the form
\begin{equation} \label{eq:frobenius homogeneous}
  \deriv[2]{y}{t} + p(t) \deriv{y}{t} + q(t) y = 0,
\end{equation}
if $p(t)$ and $q(t)$ can be expressed as power series. However, if
$p(t)$ and $q(t)$ can be expressed as series of the form\footnote{To
  find a series expansion of this type for $p(t)$, divide the power
  series for $tp(t)$ by $t$. For $q(t)$, divide the power series for
  $t^2q(t)$ by $t^2$.}:
\begin{align*}
  p(t) &= \frac{p_0}{t} + p_1 + p_2t + p_3t^2 + \cdots
        = \sum_{n=0}^\infty p_n t^{n-1} \\
  q(t) &= \frac{q_0}{t^2} + \frac{q_1}{t} + q_2 + q_3t + \cdots
        = \sum_{n=0}^\infty q_n t^{n-2},
\end{align*}
then we may still be able to find a series solution, using what is
called the method of Frobenius. The essence of this method is to
account for the singularity at $t = 0$ by multiplying our guess for
$y(t)$ by an arbitrary factor of $t^r$.\footnote{This is inspired by
  the fact that fractional powers of $t$ cannot, in general, be
  expressed as power series centered at $t = 0$.} That is, we assume
\begin{equation*}
  y(t) = t^r \sum_{n=0}^\infty a_n t^n = \sum_{n=0}^\infty a_n t^{n + r}.
\end{equation*}
With a solution of this form, we may run into trouble with negative
$t$---for instance, obtaining terms of the form $(-1)^{1/2}$. However,
restricting our analysis to $t > 0$ is not really a problem since the
differential equation is not even defined for $t = 0$.\footnote{To
  find solutions for $t < 0$ instead, simply use the change of
  variables $x = -t$.}

Differentiating, we find:
\begin{align*}
  \deriv{y}{t} &= \sum_{n=0}^\infty (n+r) a_n t^{n+r-1} \\
  \deriv[2]{y}{t} &= \sum_{n=0}^\infty (n+r)(n+r-1) a_n t^{n+r-2}.
\end{align*}
Plugging the series for $p(t)$, $q(t)$, $y(t)$, $y'(t)$, and $y''(t)$
into \eq{frobenius homogeneous} gives
\begin{equation*}
    \sum_{n=0}^\infty (n+r)(n+r-1) a_n t^{n+r-2}
  + \sum_{n=0}^\infty p_n t^{n-1} \sum_{n=0}^\infty (n+r) a_n t^{n+r-1}
  + \sum_{n=0}^\infty q_n t^{n-2} \sum_{n=0}^\infty       a_n t^{n+r}
  = 0.
\end{equation*}
Using the same property of multiplication of power series that we used
in section \ref{subsec:power series}, we obtain
\begin{equation*}
    \sum_{n=0}^\infty (n+r)(n+r-1) a_n t^{n+r-2}
  + \sum_{n=0}^\infty \sum_{k=0}^n p_{n-k} (k+r) a_k t^{n+r-2}
  + \sum_{n=0}^\infty \sum_{k=0}^n q_{n-k} a_k t^{t+r-2}
  = 0.
\end{equation*}
Grouping then gives
\begin{equation*}
  \sum_{n=0}^\infty \left\{
      (n+r)(n+r-1) a_n
    + \sum_{k=0}^n \Big[p_{n-k} (k+r) a_k + q_{n-k} a_k\Big]
  \right\} t^{n+r-2} = 0.
\end{equation*}
Now we will pull out the last term of the inner summation:
\begin{equation*}
  \sum_{n=0}^\infty \left\{
      \Big[(n+r)(n+r-1) + p_0 (n+r) + q_0\Big] a_n
    + \sum_{k=0}^{n-1} \Big[p_{n-k} (k+r) a_k + q_{n-k} a_k\Big]
  \right\} t^{n+r-2} = 0.
\end{equation*}
Finally, for convenience we introduce the notation $F(r) = r(r-1) +
p_0r + q_0$, which gives:
\begin{equation} \label{eq:frobenius lhs}
  \sum_{n=0}^\infty \left\{
      F(n + r) a_n
    + \sum_{k=0}^{n-1} \Big[p_{n-k} (k+r) a_k + q_{n-k} a_k\Big]
  \right\} t^{n+r-2} = 0.
\end{equation}
Setting the coefficient of each power of $t$ to zero, we find that for
every $n \geq 0$
\begin{equation} \label{eq:frobenius coefficients unsolved}
    F(n+r) a_n
  = -\sum_{k=0}^{n-1} \Big[p_{n-k} (k+r) a_k + q_{n-k} a_k\Big].
\end{equation}
For $n = 0$ we have
\begin{equation*}
  F(r) a_0 = 0.
\end{equation*}
We may assume that $a_0 \neq 0$, because if $a_0 = 0$ then we can just
relabel $a_1$ as $a_0$, $a_2$ as $a_1$, and so on, and increase $r$ by
$1$, to give a solution with $a_0 \neq 0$. Thus we have
\begin{equation*}
  F(r) = 0.
\end{equation*}
This is called the \term{indicial equation} of \eq{frobenius
  homogeneous}, and it has roots given by
\begin{align} \label{eq:indicial equation roots}
     r_1 = \frac{-(p_0 - 1) + \sqrt{(p_0 - 1)^2 - 4q_0}}{2},
  && r_2 = \frac{-(p_0 - 1) - \sqrt{(p_0 - 1)^2 - 4q_0}}{2}.
\end{align}
From here, generally speaking, there are three cases to consider. The
first is when $r_1 \neq r_2$ and $r_1 - r_2$ is not an integer; the
second is when $r_1 = r_2$; and the third is when $r_1 \neq r_2$ and
$r_1 - r_2$ is an integer.

\subsubsection{Roots Not Differing by an Integer}
\label{subsubsec:frobenius roots not differing by an integer}

Recall that \eq{frobenius coefficients unsolved} gives us the indicial
equation in the case of $n = 0$. For $n \geq 1$, we may solve it for
$a_n$ to find:
\begin{equation} \label{eq:frobenius coefficients}
  a_n(r) = -\left\{\frac{
      \displaystyle \sum_{k=0}^{n-1}
        \Big[p_{n-k} (k+r) + q_{n-k}\Big] a_k
    }{
      \displaystyle \vphantom{\sum_0^0} F(n + r)
    }\right\}.
\end{equation}
We are assured that $F(n + r)$ is never zero if $r$ is a root of
$F(r)$, because in this case the two roots of $F(r)$ do not differ by
an integer. Also, we have introduced the notation $a_n(r)$ to
emphasize that the value of each $a_n$ depends on the choice of $r$.

The value is $a_0$ is arbitrary; we can select any nonzero value for
it. Once we do, however, the values of $a_1(r), a_2(r), \ldots$ are
determined by \eq{frobenius coefficients}. Two linearly independent
solutions of \eq{frobenius homogeneous} are then given by:
\begin{align*}
  y_1(t) &= t^{r_1} \sum_{n=0}^\infty a_n(r_1) t^n \\
  y_2(t) &= t^{r_2} \sum_{n=0}^\infty a_n(r_2) t^n.
\end{align*}
If $r_1$ and $r_2$ are complex, then in theory two linearly
independent solutions are given by:
\begin{align*}
  y_1(t) &= \Re{t^{r_1} \sum_{n=0}^\infty a_n(r_1) t^n} \\
  y_2(t) &= \Im{t^{r_1} \sum_{n=0}^\infty a_n(r_2) t^n}.
\end{align*}
Practically speaking, these computations may be extremely difficult.

\subsubsection{Equal Roots}
\label{subsubsec:frobenius equal roots}

As we did in section \ref{subsubsec:constant coefficients equal
  roots}, we define the operator $L$ as
\begin{equation*}
  L\big[y\big] = \deriv[2]{y}{t} + p(t) \deriv{y}{t} + q(t) y.
\end{equation*}
Our derivation of \eq{frobenius lhs} illustrates that
\begin{equation*}
    L\left[t^r \sum_{n=0}^\infty a_n t^n\right]
  = \sum_{n=0}^\infty \left\{
      F(n + r) a_n + \sum_{k=0}^{n-1}
        \Big[p_{n-k} (k+r) a_k + q_{n-k} a_k\Big]
    \right\} t^{n+r-2}.
\end{equation*}
Now, if we let $a_0$ be an arbitrary, nonzero constant (independent of
the value of $r$) and define the coefficients $a_1, a_2, \ldots$
according to \eq{frobenius coefficients}, then this equation reduces
to
\begin{equation*}
  L\left[t^r \sum_{n=0}^\infty a_n(r) t^n\right] = F(r) a_0 t^{r-2}.
\end{equation*}
Since $r = r_1$ is a root of $F(r)$, we see that
\begin{equation*}
  L\left[t^{r_1} \sum_{n=0}^\infty a_n(r_1) t^n\right] = 0,
\end{equation*}
which shows that $y_1(t) = t^{r_1} \sum_{n=0}^\infty a_n(r_1) t^n$ is
a solution of \eq{frobenius homogeneous}. However, we already knew
this. To find a second solution, first notice that just as in section
\ref{subsubsec:constant coefficients equal roots} we can use the
equality of mixed partials to conclude
\begin{equation*}
  \pderiv{}{r} L\big[y\big] = L\left[\pderiv{y}{r}\right].
\end{equation*}
Now observe that because $r = r_1$ is a double root of $F(r) = r(r -
1) + p_0r + q_0$, we can factor the latter expression to $(r -
r_1)^2$. Thus,
\begin{equation*}
  L\left[t^r \sum_{n=0}^\infty a_n(r) t^n\right] = (r - r_1)^2 a_0 t^{r-2}.
\end{equation*}
Differentiating both sides of this relation with respect to $r$ gives:
\begin{align*}
     L\left[\pderiv{}{r} t^r \sum_{n=0}^\infty a_n(r) t^n\right]
  &= \pderiv{}{r} \bigg[(r - r_1)^2 a_0 t^{r-2}\bigg] \\
     L\left[(\ln t) t^r \sum_{n=0}^\infty a_n(r)
   + t^r \sum_{n=0}^\infty a_n'(r)\right]
  &= 2(r - r_1) a_0 t^{r-2} + (r - r_1)^2 a_0 (\ln r) t^{r-2}.
\end{align*}
Finally, by letting $r = r_1$, we see that
\begin{equation*}
    L\left[(\ln t) t^{r_1} \sum_{n=0}^\infty a_n(r_1)
  + t^{r_1} \sum_{n=0}^\infty a_n'(r_1)\right]
  = 0,
\end{equation*}
which means that
\begin{equation*}
    y_2(t)
  = (\ln t) t^{r_1} \sum_{n=0}^\infty a_n(r_1)
  + t^{r_1} \sum_{n=0}^\infty a_n'(r_1)
\end{equation*}
is a second solution of \eq{frobenius homogeneous}.

\subsubsection{Roots Differing by an Integer}
\label{subsubsec:frobenius roots differing by an integer}

As we have written them in \eq{indicial equation roots}, $r_1$ is
greater than $r_2$. Hence, for all $n \geq 1$, it follows that $F(n +
r_1) \neq 0$. Thus we can obtain one solution
\begin{equation*}
  y_1(t) = t^{r_1} \sum_{n=0}^\infty a_n(r_1) t^n
\end{equation*}
via the methods of section \ref{subsubsec:frobenius roots not
  differing by an integer}. However, if we try to obtain a second
solution of the form $t^{r_2} \sum_{n=0}^\infty a_n(r_2) t^n$, we will
run into trouble because $F(r_2 + n) = 0$ for $n = r_1 - r_2$. If we
are lucky, then for $n = r_1 - r_2$, we will have
\begin{equation} \label{eq:frobenius lucky}
  \sum_{k=0}^{n-1} \Big[p_{n-k} (k+r) a_k + q_{n-k} a_k\Big] = 0.
\end{equation}
In that case, \eq{frobenius coefficients unsolved} tells us that we
can pick any value for $a_n$. Then we can proceed onwards using the
methods of section \ref{subsubsec:frobenius roots not differing by an
  integer}, and we will obtain a second solution of the form
\begin{equation*}
  y_2(t) = t^{r_2} \sum_{n=0}^\infty a_n(r_2) t^n.
\end{equation*}
On the other hand, if we are not lucky enough to have \eq{frobenius
  lucky} hold, then we cannot find a solution of this form. Referring
to a more advanced text on differential equations would show that in
this case there is a second solution of the form
\begin{align*}
    y_2(t)
  = (\ln t) t^{r_2} \sum_{n=0}^\infty a_n(r_2)
  + t^{r_2} \sum_{n=0}^\infty a_n'(r_2),
\end{align*}
where $a_0(r)$, which was previously an arbitrary constant independent
of $r$, must now be equal to $r - r_2$.

\subsection{Laplace Transforms}
\label{subsec:laplace}

In this section we discuss a unique method for solving nonhomogeneous
second-order linear differential equations of the form
\begin{equation} \label{eq:laplace nonhomogeneous}
  a \deriv[2]{y}{t} + b \deriv{y}{t} + c y = f(t)
\end{equation}
for which we are given initial conditions
\begin{align*}
  y(0) &= y_0 \\
  y'(0) &= y'_0.
\end{align*}
If the initial conditions are given at $t = t_0$, then use the
transformations $\tau = t - t_0$ and $y(t) = y(\tau + t_0) = z(\tau)$,
so that the differential equation becomes
\begin{equation*}
  a \deriv[2]{z}{\tau} + b \deriv{z}{\tau} + c z = f(\tau + t_0)
\end{equation*}
and the initial conditions become
\begin{align*}
  z(0) &= y_0 \\
  z'(0) &= y'_0.
\end{align*}

This method depends upon an operator $\mathcal{L}$ called the
\term{Laplace transform}. Every function $f(t)$ has a Laplace
transform $F(s)$, which is given by
\begin{equation*}
  F(s) = \L{f(t)} = \int_0^\infty e^{-st} f(t) \,dt
\end{equation*}
The reason the Laplace transform is useful in solving differential
equations is the following property, which we derive using integration
by parts with $u = e^{-st}$, $du = -s e^{-st} \,dt$, $dv = f'(t)
\,dt$, $v = f(t)$:
\begin{equation*}
    \L{y'(t)}
  = \int_0^\infty e^{-st} y'(t) \,dt
  = \bigg[e^{-st} y(t)\bigg]_{t=0}^\infty
  + s \int_0^\infty e^{-st} y(t) \,dt
  = -y(0) + s \L{y(t)}.
\end{equation*}
Applying this property twice, we obtain:
\begin{equation*}
  \L{y''(t)} = -y'(0) + s \L{y'(t)} = -y'(0) - s y(0) + s^2 \L{y(t)}.
\end{equation*}
Thus, taking Laplace transforms of both sides of \eq{laplace
  nonhomogeneous} gives us
\begin{align*}
     -a y'(0) - as y(0) + as^2 Y(s) - b y(0) + bs Y(s) + c Y(s)
  &= F(s) \\
     \big[as^2 + bs + c\big] Y(s) - a y_0 s - y'_0 - b y_0
  &= F(s),
\end{align*}
and we can then solve algebraically for $Y(s)$:
\begin{equation*}
  Y(s) = \frac{F(s) + ay_0s + ay'_0 + by_0}{as^2 + bs + c}.
\end{equation*}
If we can find a function $y(t)$ such that $\L{y(t)} = Y(s)$, which is
called the \term{inverse Laplace transform} of $Y(s)$ and is denoted
$\Li{Y(s)}$, then this function is a solution of the original
initial-value problem \eq{laplace nonhomogeneous}.

Typically, tables of Laplace transforms are used to compute $\L{f(t)}$
and $\Li{Y(s)}$.

\subsection{Transformation to System}
\label{subsec:transformation to system}

Given the general homogeneous second-order linear differential
equation
\begin{equation*}
  y''(t) + p(t) y'(t) + q(t) y(t) = 0,
\end{equation*}
we can define
\begin{equation*}
  \vec y(t) = \Vect{y(t)}{y'(t)}.
\end{equation*}
It then follows that
\begin{align*}
  \vec y'(t) &= \Vect{y'(t)}{y''(t)} \\
             &= \Vect{y'(t)}{-p(t)y'(t) - q(t)y(t)} \\
             &= y(t) \Vect{0}{-q(t)} + y'(t) \Vect{1}{-p(t)} \\
             &= \mat[c]{0 & 1 \\ -q(t) & -p(t)} \Vect{y(t)}{y'(t)} \\
             &= \mat[c]{0 & 1 \\ -q(t) & -p(t)} \vec y(t).
\end{align*}
That is, we can transform a homogeneous second-order linear
differential equation to a system of two homogeneous first-order
linear differential equations.

\subsection{What to Do in General}
\label{subsec:second order strategy}

The following strategy is advisable:
\begin{enumerate}
\item If the equation is missing a $y$ term, use the strategy of
  section \ref{subsec:direct reduction}.
\item If the equation is linear and homogeneous and has constant
  coefficients, use the strategy of section \ref{subsec:constant
    coefficients}.
\item If the equation is linear and homogeneous and has nonconstant
  coefficients, and you already know one solution, use the strategy of
  section \ref{subsec:reduction of order} to find another.
\item If the equation is linear and homogeneous and has nonconstant
  coefficients that can be expanded as power series, use the strategy
  of section \ref{subsec:power series}.
\item If the equation is linear and homogeneous and has nonconstant
  coefficients that cannot be expanded as power series but can be
  expanded as an appropriate modified power series, use the strategy
  of section \ref{subsec:frobenius}.
\item If the equation is linear and nonhomogeneous, has constant
  coefficients, and has a right-hand side of the appropriate form, use
  the strategy of section \ref{subsec:constant coefficients} to find
  the general solution of the corresponding homogeneous equation and
  then use the strategy of section \ref{subsec:judicious guessing}.
\item If the equation is linear and nonhomogeneous and has constant
  coefficients but does not have a right-hand side admitting of the
  method of judicious guessing, either (i) use the strategy of section
  \ref{subsec:constant coefficients} to find the general solution of
  the corresponding homogeneous equation and then use the strategy of
  section \ref{subsec:variation of parameters}, or (ii) use the
  strategy of section \ref{subsec:laplace}.
\end{enumerate}

\section{Higher-Order Differential Equations}
\label{sec:higher order}

\subsection{Characterization of Higher-Order Equations}
\label{subsec:higher order characterization}

An \term{$n$th-order differential equation} is an equation of the form
\begin{equation*}
    \deriv[n]{y}{t}
  = f\left(t, y, \deriv{y}{t}, \ldots, \deriv[n-1]{y}{t}\right).
\end{equation*}
If $n$ additional equations of the form
\begin{align*}
  y(t_0) &= y_0 \\
  y'(t_0) &= y'_0 \\
         &\cvdots \\
  y^{(n-1)}(t_0) &= y^{(n-1)}_0
\end{align*}
are also given, then the $n+1$ equations together are called an
\term{initial-value problem}. Every initial-value problem has a unique
solution; however, if initial conditions are not provided, the general
solution will have $n$ arbitrary constants.\footnote{This requires
  that $f$ be sufficiently well-behaved.} An $n$th-order differential
equation of the form
\begin{equation} \label{eq:higher order nonhomogeneous}
    \deriv[n]{y}{t}
  + p_{n-1}(t) \deriv[n-1]{y}{t}
  + \cdots + p_1(t) \deriv{y}{t}
  + p_0(t) y
  = g(t)
\end{equation}
is called \term{linear}. If $g(t) = 0$ then the equation is of the
form
\begin{equation} \label{eq:higher order homogeneous}
    \deriv[n]{y}{t}
  + p_{n-1}(t) \deriv[n-1]{y}{t}
  + \cdots
  + p_1(t) \deriv{y}{t}
  + p_0(t) y = 0
\end{equation}
and it is called \term{homogeneous}; otherwise, it is called
\term{nonhomogeneous}.

A set of $n$ functions $y_1(t), y_2(t), \ldots, y_n(t)$ is called
\term{linearly independent} if there is no set of numbers $c_1, c_2,
\ldots, c_n$, not all zero, such that
\begin{equation*}
  c_1y_1(t) + c_2y_2(t) + \cdots + c_n y_n(t) = 0
\end{equation*}
for all $t$.\footnote{Three consequences of this definition are that
  none of the functions may be the zero function, that none of the
  functions may be a scalar multiple of another, and that none of the
  functions may be a linear combination of the others.} The
\term{Wronskian} of $n$ functions $y_1(t), y_2(t), \ldots, y_n(t)$ is
the quantity
\begin{equation*}
  \W{y_1, y_2, \ldots, y_n} = \dmat[c]{
    y_1(t) & y_2(t) & \cdots & y_n(t) \\
    y_1'(t) & y_2'(t) & \cdots & y_n'(t) \\
    \vdots & \vdots & \ddots & \vdots \\
    y_1^{(n-1)}(t) & y_2^{(n-1)}(t) & \cdots & y_n^{(n-1)}(t)}.
\end{equation*}
The Wronskian of $n$ linearly independent solutions of \eq{higher
  order homogeneous} is nonzero for all $t$, while the Wronskian of
$n$ linearly dependent solutions of \eq{higher order homogeneous}
solutions of \eq{higher order homogeneous} is zero for all $t$.

If $y_1(t), y_2(t), \ldots, y_n(t)$ are $n$ linearly independent
solutions of \eq{higher order homogeneous}, then the general solution
to \eq{higher order homogeneous} is
\begin{equation*}
  y(t) = C_1y_1(t) + C_2y_2(t) + \cdots + C_n y_n(t).
  \footnote{This requires that the functions $p_0(t), p_1(t), \ldots,
    p_n(t)$ are sufficiently well-behaved.}
\end{equation*}
If $y_1(t), y_2(t), \ldots, y_n(t)$ are $n$ linearly independent
solutions of \eq{higher order homogeneous} and $\psi(t)$ is a solution
of \eq{higher order nonhomogeneous}, then the general solution to
\eq{higher order nonhomogeneous} is
\begin{equation*}
  y(t) = C_1y_1(t) + C_2y_2(t) + \cdots + C_n y_n(t) + \psi(t).
  \footnote{This requires that the functions $p_0(t), p_1(t), \ldots,
    p_n(t), g(t)$ are sufficiently well-behaved.}
\end{equation*}

\subsection{Direct Reduction of Order}
\label{subsec:higher order direct reduction}

If $y$ does not appear in an $n$th-order differential equation, i.e.
it is of the form
\begin{equation*}
  \deriv[n]{y}{t} = f\left(
    t,
    \deriv{y}{t},
    \deriv[2]{y}{t},
    \ldots,
    \deriv[n-1]{y}{t}
  \right),
\end{equation*}
then letting $v = \fderiv{y}{t}$ converts it to an $(n-1)$th-order
equation:
\begin{equation*}
  \deriv[n-1]{v}{t} = f\left(
    t,
    v,
    \deriv{v}{t},
    \ldots,
    \deriv[n-2]{v}{t}
  \right).
\end{equation*}
$n-1$ constants of integration are introduced in the solution of this
$(n-1)$th-order equation, and an $n$th is introduced in recovering
$y$:
\begin{equation*}
  y(t) = \int v(t) \,dt.
\end{equation*}
In the context of an initial-value problem, the initial conditions
\begin{align*}
  y'(t_0) &= y'_0 \\
  y''(t_0) &= y''_0 \\
          &\cvdots \\
  y^{(n-1)}(t_0) &= y^{(n-1)}_0
\end{align*}
are used in the solution of the $(n-1)$th-order equation and the
initial condition $y(t_0) = y_0$ is used in recovering $y$.

\subsection{Reduction of Order}
\label{subsec:higher order reduction of order}

We rewrite \eq{higher order homogeneous} as
\begin{equation} \label{eq:higher order reduction of order}
  \sum_{n=0}^N \left[p_n(t) \deriv[n]{y}{t}\right] = 0
\end{equation}
by replacing $n$ with $N$ and letting $p_N(t) = 1$. Analogously to our
work in section \ref{subsec:reduction of order}, suppose that we have
one solution $y_1(t)$ of \eq{higher order reduction of order} and seek
$N - 1$ more linearly independent solutions. We assume that any other
solutions will be of the form
\begin{equation*}
  y(t) = y_1(t)v(t).
\end{equation*}
Differentiating, we find that
\begin{align*}
     \deriv{y}{t}
  &= \deriv{y_1}{t} v + y_1 \deriv{v}{t} \\
     \deriv[2]{y}{t}
  &=       \deriv[2]{y_1}{t} v
     + 2   \deriv   {y_1}{t} \deriv{v}{t}
     + y_1 \deriv[2]{v  }{t} \\
     \deriv[3]{y}{t}
  &=     \deriv[3]{y_1}{t} v
     + 3 \deriv[2]{y_1}{t} \deriv   {v}{t}
     + 3 \deriv   {y_1}{t} \deriv[2]{v}{t}
     +             y_1     \deriv[3]{v}{t},
\end{align*}
and in general
\begin{equation*}
    \deriv[n]{y}{t}
  = \sum_{k=0}^n \left[
      \binom{n}{k} \deriv[n-k]{y_1}{t} \deriv[k]{v}{t}
    \right].
\end{equation*}
Hence, from \eq{higher order reduction of order} we have
\begin{equation*}
  \sum_{n=0}^N \left\{p_n(t) \sum_{k=0}^n \left[
    \binom{n}{k} \deriv[n-k]{y_1}{t} \deriv[k]{v}{t}
  \right]\right\} = 0.
\end{equation*}
Pulling out each of the $k = 0$ terms gives us
\begin{equation*}
    v \sum_{n=0}^N \left [p_n(t) \deriv[n-k]{y_1}{t}\right]
  +   \sum_{n=0}^N \left\{p_n(t) \sum_{k=1}^n \left[
        \binom{n}{k} \deriv[n-k]{y_1}{t} \deriv[k]{v}{t}
      \right]\right\} = 0,
\end{equation*}
and \eq{higher order reduction of order} tells us we can cancel the
left-hand term. Using the strategy of section \ref{subsec:higher order
  direct reduction}, we let $w = \fderiv{v}{t}$ and obtain
\begin{equation} \label{eq:reduced order}
  \sum_{n=0}^N \left\{p_n(t) \sum_{k=1}^n \left[
    \binom{n}{k} \deriv[n-k]{y_1}{t} \deriv[k-1]{w}{t}
  \right]\right\},
\end{equation}
which is a homogeneous $(n-1)$th-order linear differential equation
for $w(t)$. Upon finding $n-1$ linearly independent solutions $v_1(t),
v_2(t), \ldots, v_{n-1}(t)$ of \eq{reduced order}, our $n$ linearly
independent solutions of \eq{higher order reduction of order} are then
given by:
\begin{align*}
  y_1(t) &= y_1(t) \\
  y_2(t) &= y_1(t)v_1(t) \\
  y_3(t) &= y_1(t)v_2(t) \\
         &\cvdots \\
  y_n(t) &= y_1(t)v_{n-1}(t).
\end{align*}

\subsection{Homogeneous Linear Equations with Constant Coefficients}
\label{subsec:higher order constant coefficients}

Consider the general homogeneous $n$th-order linear differential
equation with constant coefficients,
\begin{equation} \label{eq:higher order constant coefficients}
    c_n \deriv[n]{y}{t}
  + c_{n-1} \deriv[n-1]{y}{t}
  + \cdots
  + c_1 \deriv{y}{t}
  + c_0 y
  = 0.
\end{equation}
As in section \ref{subsec:constant coefficients}, we assume that
solutions are of the form $y(t) = e^{rt}$. In that case,
$\fderiv[k]{y}{t} = r^k e^{rt}$ and we obtain
\begin{equation*}
  (c_n r^n + c_{n-1} r^{n-1} + \cdots + c_1 r + c_0) e^{rt} = 0,
\end{equation*}
or equivalently
\begin{equation*}
  c_n r^n + c_{n-1} r^{n-1} + \cdots + c_1 r + c_0 = 0.
\end{equation*}
This equation has $n$ roots (not necessarily real and not necessary
distinct), which give us the solutions
\begin{align*}
  y_1(t) &= e^{r_1t} \\
  y_2(t) &= e^{r_2t} \\
         &\cvdots \\
  y_n(t) &= e^{r_n t}.
\end{align*}
We now address the cases of complex roots and repeated roots.

Suppose that one of the roots $r_i$ is complex; that is, $r_i =
\lambda + i \mu$ for real numbers $\lambda$ and $\mu$. It then follows
that there is another root $r_j$ such that $r_j = \lambda - i
\mu$.\footnote{This result is called the complex conjugate root
  theorem and is not too difficult to prove.} We then have two
complex-valued solutions of \eq{higher order constant coefficients}:
\begin{align*}
  y_i(t) &= e^{(\lambda + i \mu)t} \\
  y_j(t) &= e^{(\lambda - i \mu)t}.
\end{align*}
In this case, we can use the reasoning of section
\ref{subsubsec:constant coefficients complex roots} to replace these
complex-valued solutions with the real-valued solutions
\begin{align*}
  y_i(t) &= e^{\lambda t} \cos(\mu t) \\
  y_j(t) &= e^{\lambda t} \sin(\mu t).
\end{align*}

Now suppose that one of the roots is repeated $m$ times (that is, it
has multiplicity $m$). Without loss of generality, we will assume that
$r_1 = r_2 = \cdots = r_m$. As we did in section
\ref{subsubsec:constant coefficients equal roots}, we define the
operator $L$ as
\begin{equation*}
    L\big[y\big]
  = c_n \deriv[n]{y}{t}
  + c_{n-1} \deriv[n-1]{y}{t}
  + \cdots
  + c_1 \deriv{y}{t}
  + c_0 y,
\end{equation*}
so that
\begin{equation*}
  L\big[e^{rt}\big] = p(r) e^{rt},
\end{equation*}
where
\begin{equation*}
  p(r) = c_n r^n + c_{n-1} r^{n-1} + \cdots + c_1 r + c_0.
\end{equation*}
As we know, since $r_1$ is a root of $p(r)$, we have
\begin{equation*}
  L\big[e^{r_1t}\big] = 0,
\end{equation*}
but this is old news. What is more interesting is that since $r_1$ is
root of $p(r)$ with multiplicity $m$, we can factor $p(r)$ into $(r -
r_1)^m q_0(r)$, where $q_0(r)$ is a polynomial of degree $n - m$. Thus
we have
\begin{equation*}
  L\big[e^{rt}\big] = q(r) (r - r_1)^m e^{rt}.
\end{equation*}
Differentiating both sides with respect to $r$ and bringing the
partial derivative $\fpderiv{}{r}$ within the operator $L$ gives us
\begin{align*}
     L\big[t e^{rt}\big]
  &=   q_0'(r)(r - r_1)^m e^{rt}
     + m q_0(r)(r - r_1)^{m-1} e^{rt}
     + t q_0(r)(r - r_1)^m e^{rt} \\
  &= \Big[
           q_0'(r)(r - r_1)
       + m q_0 (r)
       + t q   (r)(r - r_1)
      \Big](r - r_1)^{m-1} e^{rt} \\
  &= q_1(r)(r - r_1)^{m-1} e^{rt},
\end{align*}
where $q_1(r)$ is another polynomial. Plugging in $r = r_1$ then shows that
\begin{equation*}
  L\big[t e^{r_1t}\big] = 0,
\end{equation*}
and thus $y_2(t) = t e^{r_1t}$ is another solution of \eq{higher order
  constant coefficients}. Differentiating a second time will give
\begin{equation*}
  L\big[t^2 e^{rt}\big] = q_2(r)(r - r_1)^{m-2} e^{rt},
\end{equation*}
which produces $y_3(t) = t^2 e^{rt}$ as a third solution of \eq{higher
  order constant coefficients}. In general, we can repeat this
reasoning until we obtain
\begin{equation*}
  L\big[t^m e^{rt}\big] = q_m(t)e^{rt},
\end{equation*}
at which point the $(r - r_1)$ term has disappeared and we can no
longer be assured that the right-hand side will be zero at $r = r_1$.
This is fine, though, because before we reach this point we obtain $m$
solutions in total:
\begin{align*}
  y_1(t) &= e^{r_1 t} \\
  y_2(t) &= te^{r_1 t} \\
  y_3(t) &= t^2 e^{r_1 t} \\
         &\cvdots \\
  y_m(t) &= t^{m-1} e^{r_1 t}.
\end{align*}

\subsection{Variation of Parameters}
\label{subsec:higher order variation of parameters}

Suppose that we have $n$ linearly independent solutions of
\begin{equation} \label{eq:higher order variation of parameters homogeneous}
    \deriv[n]{y}{t}
  + p_{n-1}(t) \deriv[n-1]{y}{t}
  + \cdots
  + p_1(t) \deriv{y}{t}
  + p_0(t) y
  = 0
\end{equation}
denoted $y_1(t), y_2(t), \ldots, y_n(t)$, and suppose furthermore that
we would like to find a particular solution $\psi(t)$ to
\begin{equation} \label{eq:higher order variation of parameters nonhomogeneous}
    \deriv[n]{y}{t}
  + p_{n-1}(t) \deriv[n-1]{y}{t}
  + \cdots + p_1(t) \deriv{y}{t}
  + p_0(t) y
  = g(t).
\end{equation}
We will assume that the solution is of the form
\begin{equation*}
  \psi(t) = u_1(t)y_1(t) + u_2(t)y_2(t) + \cdots + u_n(t)y_n(t).
\end{equation*}
Similarly to our work in section \ref{subsec:variation of parameters},
we have replaced the problem of finding one function $\psi(t)$ with
the problem of finding $n$ functions $y_1(t), \ldots, y_n(t)$; in a
sense, we have added $n-1$ additional degrees of freedom to our
problem and are therefore justified in adding $n-1$ additional
constraints on $u_1(t), \ldots, u_n(t)$ without making it so that
there are no solutions. Differentiating, we find that
\begin{equation*}
  \deriv{\psi}{t} =   u_1'(t)y_1 (t)
                    + u_2'(t)y_2 (t)
                    + \cdots
                    + u_n'(t)y_n (t)
                    + u_1 (t)y_1'(t)
                    + u_2 (t)y_2'(t)
                    + \cdots
                    + u_n (t)y_n'(t).
\end{equation*}
To avoid second-order derivatives of $u_1(t), \ldots, u_n(t)$ from
appearing when we differentiate this expression, we make the arbitrary
assumption that
\begin{equation*}
  u_1'(t)y_1(t) + u_2'(t)y_2(t) + \cdots + u_n'(t)y_n(t) = 0,
\end{equation*}
so that
\begin{align*}
     \deriv{\psi}{t}
  &= u_1(t)y_1'(t) + u_2(t)y_2'(t) + \cdots + u_n(t)y_n'(t) \\
     \deriv[2]{\psi}{t}
  &=   u_1'(t)y_1'(t)
     + u_2'(t)y_2'(t)
     + \cdots
     + u_n'(t)y_n'(t)
     + u_1(t)y_1''(t)
     + u_2(t)y_2''(t)
     + \cdots
     + u_n(t)y_n''(t).
\end{align*}
We now make the assumption that
\begin{equation*}
  u_1'(t)y_1'(t) + u_2'(t)y_2'(t) + \cdots + u_n'(t)y_n'(t) = 0,
\end{equation*}
and proceeding in this manner we find that
\begin{align*}
     \deriv{\psi}{t}
  &= u_1(t)y_1'(t) + u_2(t)y_2'(t) + \cdots + u_n(t)y_n'(t) \\
     \deriv[2]{\psi}{t}
  &= u_1(t)y_1''(t) + u_2(t)y_2''(t) + \cdots + u_n(t)y_n''(t) \\
  &\cvdots \\
     \deriv[n-1]{\psi}{t}
  &=   u_1(t)y_1^{(n-1)}(t)
     + u_2(t)y_2^{(n-1)}(t)
     + \cdots
     + u_n(t)y_n^{(n-1)}(t),
\end{align*}
assuming that
\begin{align*}
     u_1'(t)y_1(t) + u_2'(t)y_2(t) + \cdots + u_n'(t)y_n(t)
  &= 0 \\
     u_1'(t)y_1'(t) + u_2'(t)y_2'(t) + \cdots + u_n'(t)y_n'(t)
  &= 0 \\
  &\cvdots \\
     u_1'(t)y_1^{(n-2)}(t)
   + u_2'(t)y_2^{(n-2)}(t)
   + \cdots
   + u_n'(t)y_n^{(n-2)}(t)
  &= 0.
\end{align*}
These $n-1$ equations are our constraints on the functions $u_1(t),
\ldots, u_n(t)$. Differentiating one final time, we find that
\begin{align*}
     \deriv[n]{\psi}{t}
  &= u_1(t)y_1^{(n)}(t)
   + u_2(t)y_2^{(n)}(t)
   + \cdots
   + u_n(t)y_n^{(n)}(t) \\
  &+ u_1'(t)y_1^{(n-1)}(t)
   + u_2'(t)y_2^{(n-1)}(t)
   + \cdots
   + u_n'(t)y_n^{(n-1)}(t).
\end{align*}
Now we can plug our formulae for $\psi, \fderiv{\psi}{t}, \ldots,
\fpderiv[n]{\psi}{t}$ into \eq{higher order variation of parameters
  nonhomogeneous}:
\begin{align*}
  &\phantom{{}+{}} u_1(t)y_1^{(n)}(t)
              +    u_2(t)y_2^{(n)}(t)
              +    \cdots
              +    u_n(t)y_n^{(n)}(t) \\
  &           +    u_1'(t)y_1^{(n-1)}(t)
              +    u_2'(t)y_2^{(n-1)}(t)
              +    \cdots
              +    u_n'(t)y_n^{(n-1)}(t) \\
  &           +    p_{n-1}(t)\Big[u_1(t)y_1^{(n-1)}(t)
              +    u_2(t)y_2^{(n-1)}(t)
              +    \cdots
              +    u_n(t)y_n^{(n-1)}(t)\Big] \\
  &           +    p_{n-2}(t)\Big[u_1(t)y_1^{(n-2)}(t)
              +    u_2(t)y_2^{(n-2)}(t)
              +    \cdots
              +    u_n(t)y_n^{(n-3)}(t)\Big] \\
  &   \cvdots[+] \\
  &           +    p_1(t)\Big[u_1(t)y_1'(t)
              +    u_2(t)y_2'(t)
              +    \cdots
              +    u_n(t)y_n'(t)\Big] \\
  &           +    p_0(t)\Big[u_1(t)y_1(t)
              +    u_2(t)y_2(t)
              +    \cdots
              +    u_n(t)y_n(t)\Big] = g(t).
\end{align*}
Rearranging then gives
\begin{align*}
  &\phantom{{}+{}}   u_1'(t)y_1^{(n-1)}(t)
                   + u_2'(t)y_2^{(n-1)}(t)
                   + \cdots
                   + u_n'(t)y_n^{(n-1)}(t) \\
  &           +    u_1(t)\Big[
                       y_1^{(n)}(t)
                     + p_{n-1}(t)y_1^{(n-1)}(t)
                     + p_{n-2}(t)y_1^{(n-2)}(t)
                     + \cdots
                     + p_1(t)y_1'(t)
                     + p_0(t)y_1(t)
                   \Big] \\
  &           +    u_2(t)\Big[
                       y_2^{(n)}(t)
                     + p_{n-1}(t)y_2^{(n-1)}(t)
                     + p_{n-2}(t)y_2^{(n-2)}(t)
                     + \cdots
                     + p_1(t)y_2'(t)
                     + p_0(t)y_2(t)
                   \Big] \\
  &   \cvdots[+] \\
  &           +    u_n(t)\Big[
                       y_n^{(n)}(t)
                     + p_{n-1}(t)y_n^{(n-1)}(t)
                     + p_{n-2}(t)y_n^{(n-2)}(t)
                     + \cdots
                     + p_1(t)y_n'(t)
                     + p_0(t)y_n(t)
                   \Big]
              = g(t),
\end{align*}
and since $y_1(t), y_2(t), \ldots, y_n(t)$ are all solutions of
\eq{higher order variation of parameters homogeneous}, we obtain
\begin{equation*}
    u_1'(t)y_1^{(n-1)}(t)
  + u_2'(t)y_2^{(n-1)}(t)
  + \cdots
  + u_n'(t)y_n^{(n-1)}(t)
  = g(t).
\end{equation*}
Combining this equation with the $n-1$ additional constraints we
derived above gives us a system of $n$ equations which we can solve
algebraically for the functions $u_1'(t), u_2'(t), \ldots, u_n'(t)$.
Using Cramer's rule to do so gives us:
\begin{equation*}
  u_k'(t) = \frac{\W[k]{u_1, u_2, \ldots, u_n}(t)}
                 {\W{u_1, u_2, \ldots, u_n}(t)},
\end{equation*}
where $W$ denotes the Wronskian as defined in section
\ref{subsec:higher order characterization} and $W_k$ denotes the
Wronskian with the $k$th column of the determinant replaced by the
column vector $\vect{0, 0, \ldots, 0, g(t)}$. Therefore, a particular
solution to \eq{higher order variation of parameters nonhomogeneous}
is given by:
\begin{equation*}
  \psi(t) = \sum_{k=0}^n \left[
    y_k(t) \int \frac{\W[k]{u_1, u_2, \ldots, u_n}(t)}
                     {\W{u_1, u_2, \ldots, u_n}(t)}
                \,dt
  \right].
\end{equation*}

\subsection{Method of Judicious Guessing}
\label{subsec:higher order judicious guessing}

We present the results for the method of judicious guessing for
higher-order differential equations without proof.

Particular solutions of
\begin{equation*}
    c_n \deriv[n]{y}{t} + \cdots + c_1 \deriv{y}{t} + c_0 y
  = a_0 + a_1 t + \cdots + a_k t^k
\end{equation*}
occur in the forms
\begin{align*}
     A_0 + A_1 t + \cdots + A_k t^k,
  && \text{if $c_0 \neq 0$}, \\
     t\Big[A_0 + A_1 t + \cdots + A_k t^k\Big],
  && \text{if $c_0 = 0$ but $c_1 \neq 0$}, \\
     t^2\Big[A_0 + A_1 t + \cdots + A_k t^k\Big],
  && \text{if $c_0 = c_1 = 0$ but $c_2 \neq 0$}, \\
  \vdots \hspace{52.25pt} && \vdots \hspace{44.5pt} \\
     t^n\Big[A_0 + A_1 t + \cdots + A_k t^k\Big],
  && \text{if $c_0 = c_1 = \cdots = c_{n-1} = 0$ but $c_n \neq 0$}.
\end{align*}
If all the coefficients $c_i$ are zero, the differential equation has
no solution, so we need not consider this case.

Considering the homogeneous equation
\begin{equation}
  \label{eq:higher order judicious guessing homogeneous}
  c_n \deriv[n]{y}{t} + \cdots + c_1 \deriv{y}{t} + c_0 y = 0,
\end{equation}
particular solutions of
\begin{equation*}
    c_n \deriv[n]{y}{t} + \cdots + c_1 \deriv{y}{t} + c_0 y
  = \Big[a_0 + a_1 t + \cdots + a_k t^k\Big]e^{\gamma t}
\end{equation*}
occur in the forms
\begin{align*}
        \Big[A_0 + A_1 t + \cdots + A_k t^k\Big]e^{\gamma t},
  && \text{if $y = e^{\gamma t}$ does not solve
           \eq{higher order judicious guessing homogeneous}}, \\
     t  \Big[A_0 + A_1 t + \cdots + A_k t^k\Big]e^{\gamma t},
  && \text{if $y = e^{\gamma t}$ solves
           \eq{higher order judicious guessing homogeneous}
           but $y = te^{\gamma t}$ does not}, \\
     t^2\Big[A_0 + A_1 t + \cdots + A_k t^k\Big]e^{\gamma t},
  && \text{if $y = te^{\gamma t}$ solves
           \eq{higher order judicious guessing homogeneous}
           but $y = t^2e^{\gamma t}$ does not}, \\
  \vdots \hspace{66.25pt} && \vdots \hspace{44.5pt} \\
     t^n\Big[A_0 + A_1 t + \cdots + A_k t^k\Big]e^{\gamma t},
  && \text{if $y = t^{n-1} e^{\gamma t}$ solves
           \eq{higher order judicious guessing homogeneous}
           but $y = t^n e^{\gamma t}$ does not}. \\
\end{align*}
It is impossible for $y = t^n e^{\gamma t}$ to be a solution of
\eq{higher order judicious guessing homogeneous}, but we left this
condition in for symmetry.

Particular solutions of
\begin{equation*}
    c_n \deriv[n]{y}{t} + \cdots + c_1 \deriv{y}{t} + c_0 y
  = \Big[a_0 + a_1 t + \cdots + a_k t^k\Big]e^{\lambda t} \cos{\mu t}
\end{equation*}
occur in the forms
\begin{align*}
    \Big[A_0 + A_1t + \cdots + A_k t^k\Big] e^{\lambda t} \cos(\mu t)
  + \Big[B_0 + B_1t + \cdots + B_k t^k\Big] e^{\lambda t} \sin(\mu t),
  \qquad \qquad \\
  \text{if $y = e^{\lambda t} \cos(\mu t)$ does not solve
        \eq{higher order judicious guessing homogeneous}}, \\
    t\Big[A_0 + A_1t + \cdots + A_k t^k\Big] e^{\lambda t} \cos(\mu t)
  + t\Big[B_0 + B_1t + \cdots + B_k t^k\Big] e^{\lambda t} \sin(\mu t),
  \qquad \qquad \\
  \text{if $y = e^{\lambda t} \cos(\mu t)$ solves
        \eq{higher order judicious guessing homogeneous}
        but $y = te^{\lambda t} \cos(\mu t)$ does not}, \\
    t^2\Big[A_0 + A_1t + \cdots + A_k t^k\Big] e^{\lambda t} \cos(\mu t)
  + t^2\Big[B_0 + B_1t + \cdots + B_k t^k\Big] e^{\lambda t} \sin(\mu t),
  \qquad \qquad \\
  \text{if $y = te^{\lambda t} \cos(\mu t)$ solves
        \eq{higher order judicious guessing homogeneous}
        but $y = t^2e^{\lambda t} \cos(\mu t)$ does not}, \\
  \vdots \hspace{239.5pt} \\
    t^n\Big[A_0 + A_1t + \cdots + A_k t^k\Big] e^{\lambda t} \cos(\mu t)
  + t^n\Big[B_0 + B_1t + \cdots + B_k t^k\Big] e^{\lambda t} \sin(\mu t),
  \qquad \qquad \\
  \text{if $y = t^{n-1}e^{\lambda t} \cos(\mu t)$ solves
        \eq{higher order judicious guessing homogeneous}
        but $y = t^n e^{\lambda t} \cos(\mu t)$ does not}.
\end{align*}
Again, it is impossible for $y = t^n e^{\gamma t} \cos(\mu t)$ to be a
solution of \eq{higher order judicious guessing homogeneous}, but we
left this condition in for symmetry. The result for
\begin{equation*}
    c_n \deriv[n]{y}{t} + \cdots + c_1 \deriv{y}{t} + c_0 y
  = \Big[a_0 + a_1 t + \cdots + a_k t^k\Big]e^{\lambda t} \sin{\mu t}
\end{equation*}
is analogous.

The discussion in section \ref{subsubsec:judicious guessing general}
also applies to higher-order equations.

\subsection{Power Series Solutions}
\label{subsec:higher order power series}

Considering the homogeneous $n$th-order linear differential equation
\begin{equation*}
    P_ n   (t) \deriv[n  ]{y}{t}
  + P_{n-1}(t) \deriv[n-1]{y}{t}
  + \cdots
  + P_1    (t) \deriv     {y}{t}
  + P_0(t)                 y
  = 0,
\end{equation*}
we suppose that solutions will be of the form
\begin{equation*}
  y(t) = \sum_{n=0}^\infty a_n t^n.
\end{equation*}
The analysis is exactly analogous to that of section \ref{subsec:power
  series}.

\subsection{Laplace Transforms}
\label{subsec:higher order laplace}

Suppose that we are given a nonhomogeneous $n$th-other linear
differential equation of the form
\begin{equation} \label{eq:higher order laplace}
    c_ n    \deriv[n  ]{y}{t}
  + c_{n-1} \deriv[n-1]{y}{t}
  + \cdots
  + c_ 1    \deriv     {y}{t}
  + c_ 0                y
  = f(t)
\end{equation}
and the initial conditions
\begin{align*}
  y        (t_0) &= y        _0 \\
  y'       (t_0) &= y'       _0 \\
                 &\cvdots       \\
  y^{(n-1)}(t_0) &= y^{(n-1)}_0.
\end{align*}
If the initial conditions are not given at $t = 0$, we may use the
transformation discussed in section \ref{subsec:laplace}.

It can then be shown, using a simple generalization of the method of
section \ref{subsec:laplace}, that a particular solution of \eq{higher
  order laplace} is given by
\begin{equation*}
  y(t) = \Li{\frac{
        \displaystyle \L{f(t)}
      + \sum_{k=0}^n \sum_{i=0}^{k-1}
          \Big[c_k s^{k-i-1} y_0^{(i)}\Big]
    }{
      \displaystyle \sum_{k=0}^n \Big[c_k s^k\Big]
    }}.
\end{equation*}

\subsection{Transformation to System}
\label{subsec:higher order transformation to system}

Given the general homogeneous $n$th-order linear differential equation
\begin{equation*}
  y^{(n)}(t) + p_{n-1}(t) y^{(n-1)}(t) + \cdots + p_1(t) y'(t) + p_0(t) y(t) = 0,
\end{equation*}
we can define
\begin{equation*}
  \vec y(t) = \mat[c]{y(t) \\ y'(t) \\ \vdots \\ y^{(n-1)}(t)}.
\end{equation*}
It then follows that
\begin{align*}
     \vec y'(t)
  &= \mat[c]{
       y' (t)     \\
       y''(t)     \\
       \vdots     \\
       y^{(n)}(t)
     } \\
  &= \mat[c]{
       y'(t)                                          \\
       \vdots                                         \\
       y^{(n-1)}(t)                                   \\
       -p_{n-1}(t) y^{(n-1)}(t) - \cdots - p_0(t)y(t)
     } \\
  &= y(t) \mat[c]{
       0       \\
       0       \\
       \vdots  \\
       -p_0(t)
     }
   + y'(t) \mat[c]{
       1       \\
       0       \\
       \vdots  \\
       -p_1(t)
     }
   + y''(t) \mat[c]{
      0       \\
      1       \\
      \vdots  \\
      -p_2(t)
     }
   + \cdots \\
  &\quad
   + y^{(n-2)}(t) \mat[c]{
       0           \\
       \vdots      \\
       0           \\
       -p_{n-2}(t)
     }
   + y^{(n-1)}(t) \mat[c]{
       0          \\
       \vdots     \\
       1          \\
       -p_{n-1}(t)
     } \\
  &= \mat[c]{
        0      &  1      &  0      &  0      & \cdots &  0          \\
        0      &  0      &  1      &  0      & \cdots &  0          \\
        0      &  0      &  0      &  1      & \cdots &  0          \\
        \vdots &  \vdots &  \vdots &  \vdots & \ddots &  \vdots     \\
        0      &  0      &  0      &  0      & \cdots &  1          \\
       -p_0(t) & -p_1(t) & -p_2(t) & -p_3(t) & \cdots & -p_{n-1}(t)
     }
     \mat[c]{
       y(t)         \\
       y'(t)        \\
       y''(t)       \\
       \vdots       \\
       y^{(n-2)}(t) \\
       y^{(n-1)}(t)
     } \\
  &= \mat[c]{
        0      &  1      &  0      &  0      & \cdots &  0          \\
        0      &  0      &  1      &  0      & \cdots &  0          \\
        0      &  0      &  0      &  1      & \cdots &  0          \\
        \vdots &  \vdots &  \vdots &  \vdots & \ddots &  \vdots     \\
        0      &  0      &  0      &  0      & \cdots &  1          \\
       -p_0(t) & -p_1(t) & -p_2(t) & -p_3(t) & \cdots & -p_{n-1}(t)
     } \vec y(t).
\end{align*}
That is, we can transform a homogeneous $n$th-order linear
differential equation to a system of $n$ homogeneous first-order
linear differential equations.

\subsection{What to Do in General}
\label{subsec:higher order strategy}

The following strategy is advisable:
\begin{enumerate}
\item If the equation is missing a $y$ term, use the strategy of
  section \ref{subsec:higher order direct reduction}.
\item If the equation is linear and homogeneous and has constant
  coefficients, use the strategy of section \ref{subsec:higher order
    constant coefficients}.
\item If the equation is linear and homogeneous and has nonconstant
  coefficients, and you already know one solution, use the strategy of
  section \ref{subsec:higher order reduction of order} to find
  another.
\item If the equation is linear and homogeneous and has nonconstant
  coefficients that can be expanded as power series, use the strategy
  of section \ref{subsec:higher order power series}.
\item If the equation is linear and nonhomogeneous, has constant
  coefficients, and has a right-hand side of the appropriate form, use
  the strategy of section \ref{subsec:higher order constant
    coefficients} to find the general solution of the corresponding
  homogeneous equation and then use the strategy of section
  \ref{subsec:higher order judicious guessing}.
\item If the equation is linear and nonhomogeneous and has constant
  coefficients but does not have a right-hand side admitting of the
  method of judicious guessing, either (i) use the strategy of section
  \ref{subsec:higher order constant coefficients} to find the general
  solution of the corresponding homogeneous equation and then use the
  strategy of section \ref{subsec:higher order variation of
    parameters}, or (ii) use the strategy of section
  \ref{subsec:higher order laplace}.
\end{enumerate}

\end{document}
