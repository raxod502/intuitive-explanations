\documentclass{myart}
\usepackage{environ}
\usepackage{enumitem}

%%%%%%%%%%%%%%%%%%%%%%%%%%%%%%%%%%%%%%%%%%%%%%%%%%%%%%%%%%%%%%%%%%%%%%%%%%%%%%%%
%%%% Hacky stuff

% change formatting of amsthms
\makeatletter
\def\th@definition{%
  \thm@notefont{}%
  \normalfont%
}
\makeatother

% macro to check for empty argument to command
\makeatletter
\def\ifemptyarg#1{%
  \if\relax\detokenize{#1}\relax%
  \expandafter\@firstoftwo
  \else
  \expandafter\@secondoftwo
  \fi}
\makeatother

%%%%%%%%%%%%%%%%%%%%%%%%%%%%%%%%%%%%%%%%%%%%%%%%%%%%%%%%%%%%%%%%%%%%%%%%%%%%%%%%
%%%% Convenience macros

% vdots centered within the width of another symbol
\newcommand{\cvdots}[1][=]{\mathrel{\makebox[\widthof{#1}]{\vdots}}}

% term being defined
\newcommand{\term}[1]{\textbf{#1}}

% name of vector
\renewcommand{\vec}[1]{\ensuremath{\mathbf{#1}}}

% name of matrix
\newcommand{\mat}[1]{\ensuremath{#1}}

% set of all n-tuples of real numbers
\newcommand{\R}[1][]{\ensuremath{\mathbb{R}^{#1}}}

% $2 \by 3$ matrix => $2 \times 3$ matrix
\newcommand{\by}{\ensuremath{\times}}

% matrix
\newcommand{\MAT}[2][r]{\ensuremath{\begin{bmatrix*}[#1]#2\end{bmatrix*}}}

% matrix
\NewEnviron{matr}[1][r]{\ensuremath{\begin{bmatrix*}[#1]\BODY\end{bmatrix*}}}

% span of a set of vectors
\DeclareMathOperator{\OpSpan}{Span}

% span of a set of vectors
\newcommand{\Span}[1]{\ensuremath{\OpSpan\{#1\}}}

% set
\newcommand{\set}[1]{\ensuremath{\{#1\}}}

% inverse of a matrix
\newcommand{\inv}{\ensuremath{^{-1}}}

% transpose of a matrix
\newcommand{\trans}{\ensuremath{^T}}

% adjugate (classical adjoint) of a matrix
\DeclareMathOperator{\adj}{adj}

% null space of a matrix
\DeclareMathOperator{\Nul}{Nul}

% column space of a matrix
\DeclareMathOperator{\Col}{Col}

% row space of a matrix
\DeclareMathOperator{\Row}{Row}

% basis for a vector space
\newcommand{\basis}[1]{\ensuremath{\mathcal{#1}}}

% rank of a matrix
\DeclareMathOperator{\rank}{rank}

% row equivalence
\newcommand{\eq}{\sim}

% used in names of change-of-coordinates matrices
\newcommand{\from}{\leftarrow}

% coordinates of a vector relative to a basis
\newcommand{\coord}[2]{\ensuremath{\left[#1\right]_{#2}}}

% change-of-coordinates matrix from a basis to the standard basis
\newcommand{\chcoordr}[1]{\ensuremath{P_{#1}}}

% change-of-coordinates matrix between two bases
\newcommand{\chcoord}[2]{\ensuremath{P_{#1 \from #2}}}

% matrix for a linear transformation relative to a basis
\newcommand{\transmat}[2]{\coord{#1}{#2}}

% matrix for a linear transformation relative to two bases
\newcommand{\transmatb}[3]{\coord{#1}{#2 \from #3}}

% sequence
\newcommand{\seq}[3][]{\ifemptyarg{#1}{\ensuremath{\{{#2}_{#3}\}}}{\ensuremath{\{{#2}_{#3}^{(#1)}\}}}}

% set of all signals
\newcommand{\Signals}{\ensuremath{\mathbb{S}}}

% \many[k]{\vec v} => $\vec v_1, \ldots, \vec v_k$
\newcommand{\many}[2][n]{\ensuremath{{#2}_1, \ldots, {#2}_{#1}}}

% complex conjugate
\newcommand{\conj}[1]{\overline{#1}}
\let\Re\relax

% real component
\DeclareMathOperator{\Re}{Re}
\let\Im\relax

% imaginary component
\DeclareMathOperator{\Im}{Im}

% dot product
\renewcommand{\dot}{\cdot}

% orthogonal complement
\newcommand{\comp}{^\bot}

% magnitude of a vector
\newcommand{\norm}[1]{\left|\left|#1\right|\right|}

% orthogonal projection of a vector onto a subspace
\DeclareMathOperator{\proj}{proj}

\theoremstyle{definition}
\newtheorem{defn}{Definition}
\newtheorem{thm}{Theorem}

%%%%%%%%%%%%%%%%%%%%%%%%%%%%%%%%%%%%%%%%%%%%%%%%%%%%%%%%%%%%%%%%%%%%%%%%%%%%%%%%
%%%% Actual document

\begin{document}

\titlepage
{Linear Algebra Summary Sheet}
{linear-algebra-summary-sheet}
{This document is a concise collection of many of the important
  theorems of linear algebra, organized so that making connections
  between different concepts and theorems is as easy as possible. The
  material here is primarily based on the textbook I used in my Linear
  Algebra class (\textit{Linear Algebra and Its Applications} by David
  C. Lay, fourth edition).}

\section{Notation}

\begin{enumerate}
\item $I$ denotes an identity matrix, the size of which is dependent
  on context.
\item If \basis B is a basis for \R[n], then \chcoordr{\basis B}
  denotes the change-of-coordinates matrix from \basis B to \R[n],
  which satisfies $\vec x = \chcoordr{\basis B}\coord{\vec x}{\basis
    B}$ for all \vec x in \R[n].
\item If \basis B and \basis C are bases for a vector space $V$, then
  \chcoord{\basis C}{\basis B} denotes the change-of-coordinates
  matrix from \basis B to \basis C, which satisfies $\coord{\vec
    x}{\basis C} = \chcoord{\basis C}{\basis B}\coord{\vec x}{\basis
    B}$ for all \vec x in $V$.
\item If \basis B is a basis for a vector space $V$ and $T: V \to V$
  is a linear transformation, then \transmat{T}{\basis B} denotes the
  matrix for $T$ relative to \basis B, or the \basis B-matrix for $T$,
  which satisfies $\coord{T(\vec x)}{\basis B} = \transmat{T}{\basis
    B}\coord{\vec x}{\basis B}$ for all \vec x in $V$.
\item If \basis B and \basis C are bases for the vectors spaces $V$
  and $W$, respectively, and $T: V \to W$ is a linear transformation,
  then \transmatb{T}{\basis C}{\basis B} denotes the matrix for $T$
  relative to \basis B and \basis C, which satisfies $\coord{T(\vec
    x)}{\basis C} = \transmatb{T}{\basis C}{\basis B}\coord{\vec
    x}{\basis C}$ for all \vec x in $V$.
\end{enumerate}

\section{Theorems}

\begin{thm}[Solution Sets of Linear Systems] \hfill
  \begin{enumerate}
  \item A linear system either is inconsistent, has exactly one
    solution, or has infinitely many solutions.
  \item Let \mat A be an $m \by n$ matrix and let \vec b be a vector
    in \R[m]. If the equation $\mat A\vec x = \vec b$ is consistent,
    then the mapping $\vec x \mapsto \vec x + \vec b$ is a bijection
    from the solution set of $\mat A\vec x = \vec 0$ to the solution
    set of $\mat A\vec x = \vec b$.
  \end{enumerate}
\end{thm}

\begin{thm}[Consistent Linear Systems]
  Let \mat A = \MAT{\vec a_1 & \cdots & \vec a_n} be an $m \by n$
  matrix, let \vec b be a vector in \R[m], let $M$ be the augmented
  matrix \MAT{\mat A & \vec b}, and let $L$ be the linear system with
  augmented matrix $M$. Then the following statements are logically
  equivalent:
  \begin{enumerate}
  \item The linear system $L$ is consistent.
  \item The last column of $M$ is not a pivot column.
  \item The echelon form of $M$ has no row of the form \MAT{0 & \cdots
      & 0 & b}, where $b \neq 0$.
  \item The equation $x_1\vec a_1 + \cdots + x_n\vec a_n = \vec b$ has
    at least one solution.
  \item The vector \vec b is a linear combination of \many{\vec a}.
  \item The vector \vec b is an element of \Span{\many{\vec a}}.
  \item The equation $\mat A\vec x = \vec b$ has at least one
    solution.
  \end{enumerate}
\end{thm}

\begin{thm}[Always-Consistent Linear Systems]
  Let \mat A = \MAT{\vec a_1 & \cdots & \vec a_n} be an $m \by n$
  matrix and let $T : \R[n] \to \R[m]$ be the linear transformation
  given by $\vec x \mapsto \mat A\vec x$. Then the following
  statements are logically equivalent:
  \begin{enumerate}
  \item For each \vec b in \R[m], the equation $\mat A\vec x = \vec b$
    has at least one solution.
  \item Each \vec b in \R[m] is a linear combination of \many{\vec a}.
  \item The vectors \many{\vec a} span \R[m].
  \item The matrix \mat A has a pivot position in every row.
  \item The linear transformation $T$ is onto (it maps \R[n] onto
    \R[m]).
  \end{enumerate}
\end{thm}

\begin{thm}[Linear Systems with Unique Solutions]
  Let \mat A = \MAT{\vec a_1 & \cdots & \vec a_n} be an $m \by n$
  matrix, let \vec b be a vector in \R[m], let $L$ be the linear
  system with augmented matrix \MAT{\mat A & \vec b}, and let $T :
  \R[n] \to \R[m]$ be the linear transformation given by $\vec x
  \mapsto \mat A\vec x$. Then the following statements are logically
  equivalent:
  \begin{enumerate}
  \item The linear system $L$ has at most one solution.
  \item The linear system $L$ either is inconsistent or has no free
    variables.
  \item The matrix \mat A has a pivot position in every column.
  \item The equation $x_1\vec a_1 + \cdots + x_n\vec a_n = \vec b$ has
    at most one solution.
  \item The equation $\mat A\vec x = \vec b$ has at most one solution.
  \item The equation $\mat A\vec x = \vec 0$ has only the trivial
    solution.
  \item The vectors \many{\vec a} are linearly independent.
  \item The linear transformation $T$ is one-to-one.
  \end{enumerate}
\end{thm}

\begin{thm}[Square Linear Systems]
  Let \mat A = \MAT{\vec a_1 & \cdots & \vec a_n} be an $n \by n$
  matrix, and let $T : \R[n] \to \R[n]$ be the linear transformation
  given by $\vec x \mapsto \mat A\vec x$. Then the following
  statements are logically equivalent:
  \begin{enumerate}
  \item The matrix \mat A is invertible.
  \item The matrix \mat A is row equivalent to the $n \by n$ identity
    matrix.
  \item The matrix \mat A has $n$ pivot positions.
  \item For each \vec b in \R[n], the equation $\mat A\vec x = \vec b$
    has at least one solution.
  \item The vectors \many{\vec a} span \R[n].
  \item The linear transformation $T$ is onto (it maps \R[n] onto
    \R[n]).
  \item For each \vec b in \R[m], the equation $\mat A\vec x = \vec b$
    has at most one solution.
  \item The equation $\mat A\vec x = \vec 0$ has only the trivial
    solution.
  \item The vectors \many{\vec a} are linearly independent.
  \item The linear transformation $T$ is one-to-one.
  \item $\det \mat A \neq 0$.
  \item The vectors \many{\vec a} are a basis for \R[n].
  \item $\Col \mat A = \R[n]$.
  \item $\Row \mat A = \R[n]$.
  \item $\rank \mat A = n$.
  \item $\Nul \mat A = \set{\vec 0}$.
  \item $\Nul \mat A\trans = \set{\vec 0}$.
  \item $\dim \Nul \mat A = 0$.
  \item $\dim \Nul \mat A\trans = 0$.
  \item $0$ is not an eigenvalue of \mat A.
  \end{enumerate}
\end{thm}

\begin{thm}[Inversion of a Matrix]
  Let \mat A be an invertible matrix. Then \MAT{\mat A & \mat I} is
  row equivalent to \MAT{\mat I & \mat A\inv}.
\end{thm}

\begin{thm}[LU Factorization]
  Let \mat A be an $m \by n$ matrix, let \mat U be an echelon form of
  \mat A obtained only by adding multiples of rows to rows below them,
  and let \mat L be an $m \by m$ matrix that is reduced to $\mat I_m$
  by the same sequence of elementary row operations that reduces \mat
  A to \mat U. Then $\mat A = \mat L \mat U$, where \mat L is lower
  triangular and \mat U is upper triangular.
\end{thm}

\begin{thm}[Properties of Linear Dependence]
  Let \set{\many[p]{\vec a}} be a set of vectors in \R[n]. Then the
  following statements are true:
  \begin{enumerate}
  \item Suppose that $p = 1$. Then the set of vectors is linearly
    dependent if and only if $\vec a_1 = \vec 0$.
  \item Suppose that $p = 2$. Then the set of vectors is linearly
    dependent if and only if at least one of the vectors is a scalar
    multiple of the other.
  \item The set of vectors is linearly dependent if and only if at
    least one of the vectors is a linear combination of the other
    vectors.
  \item If the set of vectors is linearly dependent and $\vec a_1 \neq
    \vec 0$, then at least one of the vectors $\vec a_2, \ldots, \vec
    a_p$ is a linear combination of the preceding vectors.
  \item If $p > n$, then the set of vectors is linearly dependent.
  \item If at least one of the vectors is \vec 0, then the set of
    vectors is linearly dependent.
  \end{enumerate}
\end{thm}

\begin{thm}[Properties of Linear Transformations] \hfill
  \begin{enumerate}
  \item A transformation $T$ is linear if and only if both $T(\vec u +
    \vec v) = T(\vec u) + T(\vec v)$ and $T(c \vec u) = c\, T(\vec u)$
    for all vectors $\vec u, \vec v$ and all scalars $c$.
  \item A transformation $T$ is linear if and only if $T(c\vec u +
    d\vec v) = c\,T(\vec u) + d\,T(\vec v)$ for all vectors $\vec u,
    \vec v$ and all scalars $c, d$.
  \item If a transformation $T$ is given by $\vec x \mapsto \mat A\vec
    x$ for some $m \by n$ matrix \mat A, then $T$ is a linear
    transformation from \R[m] to \R[n].
  \item Let $T : \R[n] \to \R[m]$ be a linear transformation. Then $T$
    is given by $\vec x \mapsto \mat A \vec x$, where $\mat A = \MAT{
      T(\vec e_1) & \cdots & T(\vec e_n) }$.
  \end{enumerate}
\end{thm}

\begin{thm}[Properties of Matrix Operations] \hfill
  \begin{enumerate}
  \item Let \mat A, \mat B, and \mat C be $m \by p$, $p \by q$, and $q
    \by n$ matrices respectively. Then:
    \begin{enumerate}
    \item $\mat A (\mat B \mat C) = (\mat A \mat B) \mat C$.
    \item $\mat A (\mat B + \mat C) = \mat A \mat B + \mat A \mat C$.
    \item $(\mat A + \mat B) \mat C = \mat A \mat C + \mat B \mat C$.
    \end{enumerate}
  \item Let \mat A and \mat B be $m \by n$ matrices. Then:
    \begin{enumerate}
    \item $(\mat A\trans)\trans = \mat A$.
    \item $(\mat A + \mat B)\trans = \mat A\trans + \mat B\trans$.
    \end{enumerate}
  \item Let \many{\mat A} be matrices such that the product $\mat A_1
    \cdots \mat A_n$ is defined. Then $(\mat A_1 \cdots \mat
    A_n)\trans = \mat A_n\trans \cdots \mat A_1\trans$.
  \item Let \many{\mat A} be $n \by n$ matrices. Then $\mat A_1 \cdots
    \mat A_n$ is invertible if and only if each of \many{\mat A} are
    invertible. If $\mat A_1 \cdots \mat A_n$ is invertible, then
    $(\mat A_1 \cdots \mat A_n)\inv = \mat A_n\inv \cdots \mat
    A_1\inv$.
  \item Let \mat A be an $n \by n$ matrix. Then $\mat A\trans$ is
    invertible if and only if \mat A is invertible. If $\mat A\trans$
    is invertible, then $(\mat A\trans)\inv = (\mat A\inv)\trans$.
  \end{enumerate}
\end{thm}

\begin{thm}[Properties of Determinants] \hfill
  \begin{enumerate}
  \item If \mat A is a triangular matrix, then $\det \mat A$ is the
    product of the entries on the main diagonal of \mat A.
  \item If \mat A is a square matrix, and the matrix \mat B is
    obtained from \mat A by performing one elementary row
    operation\footnote{The same applies to elementary column
      operations.}, then:
    \begin{enumerate}
    \item If a multiple of one row of \mat A is added to another row,
      then $\det \mat B = \det \mat A$.
    \item If two rows of \mat A are interchanged, then $\det \mat B =
      -\det \mat A$.
    \item If one row of \mat A is multiplied by $k$, then $\det \mat B
      = k \det \mat A$.
    \end{enumerate}
  \item If \mat A and \mat B are square matrices, then $\det \mat
    A\mat B = (\det \mat A)(\det \mat B)$.
  \item If \mat A is invertible, then $\det \mat A$ is $(-1)^n$
    multiplied by the product of the pivots in any echelon form of
    \mat A, where $n$ is the number of row interchanges necessary to
    reduce \mat A to echelon form.
  \item If \mat A is an $n \by n$ matrix, \vec b is a vector in \R[n],
    $\det \mat A \neq 0$, and $\mat A\vec x = \vec b$, then
    \begin{equation*}
      \vec x = \frac{1}{\det \mat A} \MAT[c]{
        \det \mat A_1(\vec b) \\ \vdots \\ \det \mat A_n(\vec b)
      },
    \end{equation*}
    where $\mat A_i(\vec b)$ denotes the matrix obtained by replacing
    column $i$ of \mat A with \vec b.
  \item If \mat A is an $n \by n$ matrix and $\det \mat A \neq 0$,
    then
    \begin{equation*}
      \mat A\inv = \frac{\adj \mat A}{\det \mat A},
    \end{equation*}
    where
    \begin{align*}
      \adj \mat A &= \MAT[c]{
                       C_{11} & C_{12} & \cdots & C_{1n} \\
                       C_{21} & C_{22} & \cdots & C_{2n} \\
                       \vdots & \vdots & \ddots & \vdots \\
                       C_{n1} & C_{n2} & \cdots & C_{nn}
                     }\trans, \\
           C_{ij} &= (-1)^{i+j}\det \mat A_{ij},
    \end{align*}
    and $\mat A_{ij}$ denotes the matrix obtained by removing row $i$
    and column $j$ from \mat A.
  \item If $\mat A = \MAT{\vec a_1 & \vec a_2}$ is a $2 \by 2$ matrix,
    then $|\det \mat A|$ is the area of the parallelogram determined
    by $\vec a_1$ and $\vec a_2$. If $\mat A = \MAT{\vec a_1 & \vec
      a_2 & \vec a_3}$ is a $3 \by 3$ matrix, then $|\det \mat A|$ is
    the volume of the parallelepiped determined by $\vec a_1$, $\vec
    a_2$, and $\vec a_3$.
  \item If \mat A is a $2 \by 2$ matrix, $T : \R[2] \to \R[2]$ is the
    linear transformation defined by $\vec x \mapsto \mat A\vec x$,
    and $S$ is a region in \R[2] with finite area, then the area of
    $T(S)$ is the area of $S$ multiplied by $|\det \mat A\,|$. If \mat
    A is a $3 \by 3$ matrix, $T : \R[3] \to \R[3]$ is the linear
    transformation defined by $\vec x \mapsto \mat A\vec x$, and $S$
    is a region in \R[3] with finite volume, then the volume of $T(S)$
    is the volume of $S$ multiplied by $|\det \mat A\,|$.
  \end{enumerate}
\end{thm}

\begin{thm}[Properties of Subspaces] \hfill
  \begin{enumerate}
  \item If \many{\vec v} are vectors in a vector space $V$, then
    \Span{\many{\vec v}} is a subspace of $V$.
  \item If \mat A is an $m \by n$ matrix, then $\Nul \mat A$ is a
    subspace of \R[n].
  \item If \mat A is an $m \by n$ matrix, then $\Col \mat A$ is a
    subspace of \R[m].
  \item If \mat A is an $m \by n$ matrix, then $\Row \mat A$ is a
    subspace of \R[n].
  \end{enumerate}
\end{thm}

\begin{thm}[Properties of Bases] \hfill
  \begin{enumerate}
  \item Let $S$ be a finite subset of a vector space and let $H =
    \OpSpan S$. If one of the vectors, \vec v, in $S$ is a linear
    combination of the others, then the set formed by removing \vec v
    from $S$ still spans $H$.
  \item Let $S \neq \set{\vec 0}$ be a subset of a vector space $V$.
    If $S$ spans $V$, then some subset of $S$ is a basis for $V$, and
    if $S$ is linearly independent, then some superset of $S$ is a
    basis for $V$.
  \item If a vector space $V$ has a basis containing $n$ vectors, then
    every basis of $V$ must contain exactly $n$ vectors. Furthermore,
    any subset of $V$ containing more than $n$ vectors must be
    linearly dependent, and any subset of $V$ containing more than $n$
    elements cannot span $V$.
  \item If $V$ is an $n$-dimensional vector space, then any subset of
    $V$ containing exactly $n$ vectors that either spans $V$ or is
    linearly independent must be a basis for $V$.
  \item If $H$ is a subspace of a finite-dimensional vector space $V$,
    then $\dim H \leq \dim V$.
  \end{enumerate}
\end{thm}

\begin{thm}[Properties of Coordinates] \hfill
  \begin{enumerate}
  \item Let \basis B be a basis with $n$ vectors for a vector space
    $V$. Then the transformation $\vec x \mapsto \coord{\vec x}{\basis
      B}$ is an isomorphism from $V$ to \R[n].
  \item Let $\basis B = \set{\many{\vec b}}$ be a basis for \R[n].
    Then $\chcoordr{\basis B} = \MAT{\vec b_1 & \cdots & \vec b_n}$.
  \item Let $\basis B = \set{\many{\vec b}}$ and \basis C be bases for
    a vector space $V$. Then $\chcoord{\basis C}{\basis B} =
    \MAT{\coord{\vec b_1}{\basis C} \cdots \coord{\vec b_n}{\basis
        C}}$.
  \item Let \basis B and \basis C be bases for a vector space $V$.
    Then $\chcoord{\basis C}{\basis B} = \left(\chcoord{\basis
        B}{\basis C}\right)\inv$.
  \item Let \basis B, \basis C, and \basis D be bases for a vector
    space $V$. Then $\chcoord{\basis D}{\basis C} \chcoord{\basis
      C}{\basis B} = \chcoord{\basis D}{\basis B}$.
  \item Let $\basis B = \set{\many{\vec b}}$ and $\basis C =
    \set{\many{\vec c}}$ be bases for \R[n]. Then \MAT{\vec c_1 &
      \cdots & \vec c_n & \vec b_1 & \cdots & \vec b_n} is row
    equivalent to \MAT{\mat I & \chcoord{\basis C}{\basis B}}.
  \item Let $\basis B = \set{\many{\vec b}}$ and $\basis C =
    \set{\many{\vec c}}$ be bases for a subspace $H$ of \R[m]. Then
    the matrix \MAT{\vec c_1 & \cdots & \vec c_n & \vec b_1 & \cdots &
      \vec b_n} is row equivalent to
    \begin{equation*}
      \begin{matr}[c]
        I_n & \chcoord{\basis C}{\basis B} \\
        0 & 0
      \end{matr}
    \end{equation*}
  \item Let $\basis B = \set{\many{\vec b}}$ be a basis for a vector
    space $V$, and let $T: V \to V$ be a linear transformation. Then
    \begin{equation*}
      \transmat{T}{\basis B} = \MAT{
        \coord{T(\vec b_1)}{\basis B} &
        \cdots &
        \coord{T(\vec b_n)}{\basis B}
      }.
    \end{equation*}
  \item Let $\basis B = \set{\many{\vec b}}$ and \basis C =
    \set{\many{\vec c}} be bases for the vector spaces $V$ and $W$,
    respectively, and let $T: V \to W$ be a linear transformation.
    Then
    \begin{equation*}
      \transmatb{T}{\basis C}{\basis B} = \MAT{
        \coord{T(\vec b_1)}{\basis C} &
        \cdots &
        \coord{T(\vec b_n)}{\basis C}
      }.
    \end{equation*}
  \item Let \mat A, \mat P, and \mat B be $n \by n$ matrices such that
    $\mat A = \mat P\mat B\mat P\inv$, let \basis B be the basis for
    \R[n] formed from the columns of \mat P, and let $T$ be the
    transformation given by $\vec x \mapsto \mat A\vec x$. Then
    $\transmat{T}{\basis B} = \mat B$. In particular, if \mat B is a
    diagonal matrix, then \basis B is the eigenvector basis and the
    diagonal entries of \mat B are the corresponding eigenvalues.
  \end{enumerate}
\end{thm}

\begin{thm}[Properties of the Null Space, Row Space, and Column Space]
  Let \mat A be an $n \by m$ matrix.
  \begin{enumerate}
  \item The pivot columns of \mat A are a basis for $\Col \mat A$,
    while the pivot rows of any echelon form of \mat A are a basis for
    $\Row \mat A$.
  \item The equalities $\Col \mat A = \Row \mat A\trans$ and $\Row
    \mat A = \Col \mat A\trans$ hold.
  \item The dimension of $\Nul \mat A$ is equal to the number of free
    variables in the equation $\mat A\vec x = \vec 0$ (or equivalently
    the number of nonpivot columns of \mat A), while the dimension of
    $\Nul \mat A\trans$ is equal to the number of free variables in
    the equation $\mat A\trans\vec x = \vec 0$ (or equivalently the
    number of nonpivot columns of \mat A\trans).
  \item The dimension of $\Col \mat A$, the dimension of $\Row \mat
    A$, the number of pivot positions in \mat A, and the number of
    pivot positions in $\mat A\trans$ are all equal.
  \item The equalities $\rank \mat A + \dim \Nul \mat A = n$ and
    $\rank \mat A + \dim \Nul \mat A\trans = m$ hold.
  \item $\Row \mat A$ and $\Nul \mat A$ are orthogonal complements,
    and $\Col \mat A$ and $\Nul \mat A\trans$ are orthogonal
    complements.
  \end{enumerate}
\end{thm}

\begin{thm}[Properties of Eigenvectors and Eigenvalues] \hfill
  \begin{enumerate}
  \item A number $\lambda$ is an eigenvalue of a matrix \mat A if and
    only if $\det(\mat A - \lambda\mat I) = 0$.
  \item The eigenvalues of a triangular matrix are the entries on its
    main diagonal, with the number of times an entry is repeated being
    equal to its multiplicity as an eigenvalue.
  \item Any set of $n$ eigenvectors corresponding to $n$ unique
    eigenvalues of a matrix is linearly independent.
  \item Let \mat A be a matrix with at least $n$ distinct eigenvalues
    \many{\lambda}. If $S_k$ is a linearly independent set of
    eigenvectors corresponding to $\lambda_k$, for $1 \leq k \leq n$,
    then the union of \many{S} is also linearly independent.
  \item The eigenspace for any given eigenvalue of a matrix is a
    subspace of \R[n].
  \item The dimension of the eigenspace for any given eigenvalue of a
    matrix cannot exceed the multiplicity of the eigenvalue.
  \item The eigenspaces for any two distinct eigenvalues of a matrix
    are disjoint except at \vec 0.
  \item If an $n \by n$ matrix is diagonalizable and \many[m]{\basis
      B} are bases for the $m$ distinct eigenvalues of \mat A, then
    the union of \many[m]{\basis B} is a basis for \R[n].
  \item If two matrices are similar, then they have the same
    eigenvalues with the same multiplicities.
  \item Let \mat A be an $n \by n$ matrix. Then the following
    statements are logically equivalent:
    \begin{enumerate}
    \item The matrix \mat A is diagonalizable.
    \item The matrix \mat A has $n$ linearly independent eigenvectors.
    \item The sum of the dimensions of the eigenspaces of \mat A
      equals $n$.
    \item The dimension of the eigenspace for each eigenvalue of \mat
      A equals the multiplicity of the eigenvalue.
    \end{enumerate}
  \item If an $n \by n$ matrix \mat A has $n$ distinct eigenvalues,
    then it is diagonalizable.
  \item If a matrix \mat A is diagonalizable, then $\mat A = \mat
    P\mat D\mat P\inv$, where the columns of \mat P are linearly
    independent eigenvectors and the diagonal entries of \mat D are
    the corresponding eigenvalues.
  \item Let \mat A be a real $2 \by 2$ matrix with a complex
    eigenvalue $a - bi$ and an associated complex eigenvector \vec v.
    Then $\mat A = \mat P\mat C\mat P\inv$, where $\mat P = \MAT{\Re
      \vec v & \Im \vec v}$,
    \begin{equation*}
      \mat C = \MAT{a & -b \\ b & a}
             = \MAT{r & 0 \\ 0 & r}
               \MAT{\cos \phi & -\sin \phi \\ \sin \phi & \cos \phi},
    \end{equation*}
    and $r$ and $\phi$ are the magnitude and complex argument of $a +
    bi$.
  \end{enumerate}
\end{thm}

\begin{thm}[Properties of Orthogonal Coordinates] \hfill
  \begin{enumerate}
  \item Let \set{\many{\vec u}} be an orthogonal basis for an inner
    product space $V$, and let \vec y be a vector in $V$. Then
    \begin{equation*}
      \vec y = \frac{\vec y \dot \vec u_1}{\vec u_1 \dot \vec u_1} \vec u_1
             + \cdots
             + \frac{\vec y \dot \vec u_n}{\vec u_n \dot \vec u_n} \vec u_n.
    \end{equation*}
  \item Let \set{\many{\vec u}} be an orthonormal basis for an inner
    product space $V$, and let \vec y be a vector in $V$. Then
    \begin{equation*}
      \vec y = (\vec y \dot \vec u_1) \vec u_1
             + \cdots
             + (\vec y \dot \vec u_n) \vec u_n.
    \end{equation*}
  \end{enumerate}
\end{thm}

\begin{thm}[Properties of Orthogonal Projections]
  Let $W$ be a subspace of \R[m], let \basis U = \set{\many{\vec u}}
  be a basis for $W$, let $\mat U = \MAT{ \vec u_1 & \cdots & \vec u_n
  }$ be an $m \by n$ matrix, and let \vec y be a vector in \R[m].
  Then:
  \begin{enumerate}
  \item There is a unique representation of \vec y in the form $\vec y
    = \vec{\hat y} + \vec z$, where \vec{\hat y} is in $W$ and \vec z
    is in $W\comp$. Typically \vec{\hat y} is denoted by $\proj_W \vec
    y$.
  \item The closest point in $W$ to \vec y is \vec{\hat y}, i.e.
    $\norm{\vec y - \vec{\hat y}} < \norm{\vec y - \vec v}$ if $\vec v
    \neq \vec{\hat y}$.
  \item The equation $\mat U\trans \mat U \vec x = \mat U\trans \vec
    y$ has a unique solution, and $\vec{\hat y} = \mat \vec x$.
  \item If \basis U is an orthogonal basis for $W$, then
    \begin{equation*}
      \vec{\hat y} = \frac{\vec y \dot \vec u_1}{\vec u_1 \dot \vec u_1}
                     \vec u_1
                   + \cdots
                   + \frac{\vec y \dot \vec u_n}{\vec u_n \dot \vec u_n}
                     \vec u_n.
    \end{equation*}
  \item If \basis U is an orthonormal basis for $W$, i.e. \mat U is an
    orthogonal matrix, then
    \begin{equation*}
      \vec{\hat y} = (\vec y \dot \vec u_1) \vec u_1
                   + \cdots
                   + (\vec y \dot \vec u_n) \vec u_n
                   = \mat U \mat U\trans \vec y.
    \end{equation*}
  \item If \basis U is an orthonormal basis for \R[m], i.e. \mat U is
    a square orthogonal matrix, then $\vec{\hat y} = \vec y$.
  \item Let
    \begin{align*}
      \vec v_1 &= \vec u_1, \\
      \vec v_2 &= \vec u_2 - \proj_{\Span{\vec v_1}} u_2, \\
               &\cvdots \\
      \vec v_n &= \vec u_n - \proj_{\Span{\many{\vec v}}} u_n.
    \end{align*}
    Then \set{\many{\vec v}} is an orthogonal basis for $W$, and
    \begin{align*}
      \Span{\vec v_1} &= \Span{\vec u_1} \\
      \Span{\vec v_1, \vec v_2} &= \Span{\vec u_1, \vec u_2} \\
                      &\cvdots \\
      \Span{\many{\vec v}} &= \Span{\many{\vec u}}.
    \end{align*}
    This method of obtaining an orthogonal basis for $W$ is called the
    Gram-Schmidt process.
  \end{enumerate}
\end{thm}

\begin{thm}[Properties of Orthogonal Matrices] \hfill
  \begin{enumerate}
  \item Let \mat U be a matrix. Then \mat U has orthonormal columns if
    and only if $\mat U \mat U\trans = \mat I$, and \mat U has
    orthonormal rows if and only if $\mat U\trans \mat U = \mat I$.
  \item Let \mat U be a square matrix. Then \mat U has orthonormal
    columns if and only if it has orthonormal rows.
  \item Let \mat U be an $m \by n$ orthogonal matrix, and let \vec x
    and \vec y be vectors in \R[m]. Then \mat U preserves lengths,
    angles, and orthogonality. That is:
    \begin{enumerate}
    \item $\norm{\mat U \vec x} = \norm{\vec x}$.
    \item $(\mat U \vec x) \dot (\mat U \vec y) = \vec x \dot \vec y$.
    \item $(\mat U \vec x) \dot (\mat U \vec y) = 0$ if and only if
      $\vec x \dot \vec y = 0$.
    \end{enumerate}
  \item Let \mat A be an $m \by n$ matrix with linearly independent
    columns, and let $\mat U = \MAT{ \vec u_1 & \cdots & \vec u_n }$
    be the $m \by n$ orthogonal matrix obtained by finding an
    orthogonal basis for $\Col \mat A$ using the Gram-Schmidt process,
    normalizing the vectors, and using them as the columns of \mat U.
    Let \mat R be the $n \by n$ matrix equal to $\mat Q\trans \mat A$.
    Also, for $1 \leq i \leq n$, if the $(i, i)$ entry of \mat R is
    negative, multiply the $i$th row of \mat R and the $i$th column of
    \mat U by $-1$. Then $\mat A = \mat P \mat Q$, and \mat R is an
    upper triangular invertible matrix with positive entries on its
    diagonal.
  \end{enumerate}
\end{thm}

\begin{thm}[Properties of Inner Products]
  Let \vec u and \vec v be vectors in an inner product space. Then:
  \begin{enumerate}
  \item $\norm{\vec u + \vec v}^2 = \norm{\vec u}^2 + \norm{\vec v}^2$
    if and only if $\vec u \dot \vec v = 0$.
  \item $|\vec u \dot \vec v| \leq \norm{\vec u} \norm{\vec v}$.
  \item $\norm{\vec u + \vec v} \leq \norm{\vec u} + \norm{\vec v}$.
  \end{enumerate}
\end{thm}

\begin{thm}[Applications] \hfill
  \begin{enumerate}
  \item Let \mat C be a square matrix whose entries are nonnegative
    and whose column sums are each less than $1$. Then $\mat I + \mat
    C + \mat C^2 + \mat C^3 + \cdots = (\mat I - \mat C)\inv$.
  \item Let $\seq[1]uk, \seq[2]uk, \ldots, \seq[n]uk$ be vectors in
    \Signals. If the matrix
    \begin{equation*}
      \begin{matr}[c]
        \seq[1]uk & \seq[2]uk & \cdots & \seq[n]uk \\
        \seq[1]u{k+1} & \seq[2]u{k+1} & \cdots & \seq[n]u{k+1} \\
        \vdots & \vdots & \ddots & \vdots \\
        \seq[1]u{k+n-1} & \seq[2]u{k+n-1} & \cdots & \seq[n]u{k+n-1}
      \end{matr}
    \end{equation*}
    is invertible for at least one value of $k$, then the vectors are
    linearly independent. Furthermore, if the vectors are solutions to
    the same homogeneous linear difference equation and the matrix is
    not invertible for any value of $k$, then the vectors are linearly
    dependent.
  \item The set of all solutions to the homogeneous linear difference
    equation
    \begin{equation*}
      y_{k+n} + a_1y_{k+n-1} + \cdots + a_{n-1}y_{k+1} + a_ny_k = 0,
    \end{equation*}
    is an $n$-dimensional subspace of \Signals.
  \item Let \seq zk be a vector in \Signals. If the nonhomogeneous
    linear difference equation
    \begin{equation*}
      y_{k+n} + a_1y_{k+n-1} + \cdots + a_{n-1}y_{k+1} + a_ny_k = z_k
    \end{equation*}
    is consistent, then the mapping $\seq yk \mapsto \seq{y_k +
      z_k}{}$ is a bijection from the solution set of the homogeneous
    linear difference equation
    \begin{equation*}
      y_{k+n} + a_1y_{k+n-1} + \cdots + a_{n-1}y_{k+1} + a_ny_k = 0
    \end{equation*}
    to the solution set of the nonhomogeneous equation.
  \item The solutions of the homogeneous linear difference equation
    \begin{equation*}
      y_{k+n} + a_1y_{k+n-1} + \cdots + a_{n-1}y_{k+1} + a_ny_k = 0
    \end{equation*}
    are given by the dynamical system $\vec x_{k+1} = \mat A\vec x_k$,
    where
    \begin{equation*}
      \vec x_k = \MAT[c]{y_k \\ y_{k+1} \\ \vdots \\ y_{k+n-1}},
      \mat A =
      \begin{matr}[c]
        0      & 1        & 0       & \cdots & 0      \\
        0      & 0        & 1       & \cdots & 0      \\
        \vdots & \vdots   & \vdots  & \ddots & \vdots \\
        0      & 0        & 0       & \cdots & 1      \\
        -a_n   & -a_{n-1} & a_{n-2} & \cdots & -a_1
      \end{matr}.
    \end{equation*}
  \item Call a vector a \term{probability vector} if its entries are
    nonnegative and sum to $1$. Let \mat P be a square matrix whose
    columns are probability vectors. Then there is a unique
    probability vector \vec q such that $\mat P\vec q = \vec q$.
    Furthermore, $\lim_{k \to \infty} \mat A^k\vec x = \vec q$ for any
    probability vector \vec x.
  \item Let \mat A be an $n \by n$ matrix with $n$ linearly
    independent eigenvectors \many{\vec v}, let \many{\lambda} be the
    corresponding eigenvalues, possibly with repetition, and let
    \many{c} be the coordinates of a vector \vec x in the eigenvector
    basis. Then
    \begin{equation*}
      \mat A^k\vec x = c_1\lambda_1^k\vec v_1 + \cdots + c_n\lambda_n^k\vec v_n.
    \end{equation*}
  \item Let \mat A be a matrix. If $\mat A = \mat P\mat D\mat P\inv$,
    with \mat D a diagonal matrix, then $\mat A^k = \mat P\mat D^k\mat
    P\inv$.
  \item If \mat A is an $n \by n$ diagonalizable matrix with $n$
    linearly independent eigenvectors \many{\vec v} and corresponding
    eigenvalues \many{\lambda}, then the general solution to the
    system of differential equations given by $\vec x'(t) = \mat A\vec
    x(t)$ is $\vec x(t) = c_1\vec v_1e^{\lambda_1t} + \cdots + c_n\vec
    v_ne^{\lambda_nt}$.
  \item Suppose \mat A is an $n \by n$ matrix, the general solution to
    the system of differential equations given by $\vec x'(t) = \mat
    A\vec x(t)$ is an arbitrary linear combination of a collection of
    linearly independent eigenfunctions $\vec y_1(t), \ldots, \vec
    y_n(t)$, and two of those solutions, say $\vec y_1(t)$ and $\vec
    y_2(t)$, are complex conjugates. If $\vec y_1(t)$ and $\vec
    y_2(t)$ are replaced by either $\Re \vec y_1(t)$ and $\Im \vec
    y_1(t)$ or $\Re \vec y_2(t)$ and $\Im \vec y_2(t)$, the collection
    of eigenfunctions will still be linearly independent and an
    arbitrary linear combination of them will still give the general
    solution to the system.
  \item Let \mat A be an $m \by n$ matrix, and let \vec b be a vector
    in \R[m]. Then \vec{\hat x} is a least-squares solution of $\mat A
    \vec x = \vec b$ if and only if $\mat A\trans \mat A \vec{\hat x}
    = \mat A\trans \vec b$.
  \item Let \mat A be an $m \by n$ matrix. Then the following
    statements are logically equivalent:
    \begin{enumerate}
    \item The equation $\mat A \vec x = \vec b$ has a unique solution
      for each \vec b in \R[m].
    \item The columns of \mat A are linearly independent.
    \item The matrix $\mat A\trans \mat A$ is invertible.
    \end{enumerate}
    When these statements are true, \mat A has a QR factorization and
    the least-squares solution is given by $\vec{\hat x} = (\mat
    A\trans \mat A)\inv \mat A\trans \vec b = \mat R\inv \mat Q\trans
    \vec b$.
  \end{enumerate}
\end{thm}

\end{document}
